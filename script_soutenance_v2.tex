\documentclass[a4paper,11pt]{article}
\usepackage[utf8]{inputenc}
\usepackage[T1]{fontenc}
\usepackage[french]{babel}
\usepackage{amsmath, amssymb}
\usepackage{geometry}
\usepackage{xcolor}
\usepackage{tcolorbox}
\tcbuselibrary{skins,breakable}
\usepackage{fancyhdr}
\usepackage{enumitem}
\usepackage{tikz}
\usepackage{booktabs}
\usepackage{hyperref}

\geometry{hmargin=2.5cm,vmargin=2.5cm}
\setlength{\headheight}{14pt}

% Couleurs pour les 4 orateurs
\definecolor{johan}{RGB}{59, 130, 246}     % Bleu - Johan
\definecolor{achille}{RGB}{16, 185, 129}   % Vert - Achille
\definecolor{lionel}{RGB}{249, 115, 22}    % Orange - Lionel
\definecolor{maxime}{RGB}{139, 92, 246}    % Violet - Maxime
\definecolor{codecolor}{RGB}{30, 41, 59}

% Boîtes pour chaque orateur
\newtcolorbox{johanbox}[1][]{
    colback=johan!5, colframe=johan,
    fonttitle=\bfseries\large, title={#1}, breakable,
    left=3mm, right=3mm
}
\newtcolorbox{achillebox}[1][]{
    colback=achille!5, colframe=achille,
    fonttitle=\bfseries\large, title={#1}, breakable,
    left=3mm, right=3mm
}
\newtcolorbox{lionelbox}[1][]{
    colback=lionel!5, colframe=lionel,
    fonttitle=\bfseries\large, title={#1}, breakable,
    left=3mm, right=3mm
}
\newtcolorbox{maximebox}[1][]{
    colback=maxime!5, colframe=maxime,
    fonttitle=\bfseries\large, title={#1}, breakable,
    left=3mm, right=3mm
}

% En-tête
\pagestyle{fancy}
\fancyhf{}
\rhead{Projet 5A-IDSA -- Script de Soutenance}
\lhead{Évaluation Thermomécanique TBC}
\rfoot{Page \thepage}

\title{\textbf{SCRIPT DE SOUTENANCE}\\[0.5cm]
\Large Évaluation Thermomécanique des Zones Critiques\\
d'Endommagement dans les Aubes de Turbines\\
Multicouches Nouvelle Génération\\[1cm]
\normalsize Projet Industriel 5A-IDSA -- ONERA / ESTACA / Safran}
\author{Johan GUITTON · Achille VOISIN · Lionel DAURIAC · Maxime LEBOEUF}
\date{26 janvier 2026}

\begin{document}

\maketitle
\thispagestyle{empty}

\vspace{1cm}

\begin{center}
\begin{tcolorbox}[colback=codecolor!10, colframe=codecolor, width=0.95\textwidth]
\centering
\textbf{RÉPARTITION DES ORATEURS -- 32 SLIDES}\\[0.5cm]
\begin{tabular}{llll}
\textcolor{johan}{\rule{1cm}{0.4cm}} \textbf{Johan} & 
\textcolor{achille}{\rule{1cm}{0.4cm}} \textbf{Achille} &
\textcolor{lionel}{\rule{1cm}{0.4cm}} \textbf{Lionel} &
\textcolor{maxime}{\rule{1cm}{0.4cm}} \textbf{Maxime}
\end{tabular}\\[0.3cm]
\textit{Durée totale : 40 minutes (10 min chacun) + 15 min questions}
\end{tcolorbox}
\end{center}

\vspace{0.5cm}

\begin{center}
\begin{tabular}{|l|c|l|c|}
\hline
\textbf{Orateur} & \textbf{Slides} & \textbf{Sections} & \textbf{Durée} \\
\hline
\textcolor{johan}{\textbf{Johan}} & 1--10 & Intro + Contexte + Objectifs & 10 min \\
\textcolor{achille}{\textbf{Achille}} & 11--19 & Méthode Spectrale + Matrices & 10 min \\
\textcolor{lionel}{\textbf{Lionel}} & 20--24 & Implémentation + Dashboard & 10 min \\
\textcolor{maxime}{\textbf{Maxime}} & 25--32 & Résultats + Validation + Conclusion & 10 min \\
\hline
\end{tabular}
\end{center}

\newpage
\tableofcontents
\newpage

%═══════════════════════════════════════════════════════════════════════════════
\section{PARTIE 1 : Contexte Industriel et Objectifs (Johan -- 10 min)}
%═══════════════════════════════════════════════════════════════════════════════

\begin{johanbox}[JOHAN -- Slide 1 : Titre (30 sec)]
\textbf{[Se lever, sourire au jury]}

<<~Bonjour à tous, merci d'être présents pour notre soutenance de projet industriel.

Je suis Johan Guitton, et avec mes collègues Achille Voisin, Lionel Dauriac et Maxime Leboeuf, nous allons vous présenter notre projet :

\textbf{``Évaluation Thermomécanique des Zones Critiques d'Endommagement dans les Aubes de Turbines Multicouches Nouvelle Génération''}

Nous avons développé une \textbf{méthode spectrale} implémentée en \textbf{Python/Streamlit}, sous l'encadrement de l'ONERA avec Aurélien Vattré, et de l'ESTACA avec Daniel Gaffié.~>>
\end{johanbox}

\begin{johanbox}[JOHAN -- Slide 2 : Structure de la Présentation (30 sec)]
<<~Voici la structure de notre présentation en \textbf{4 parties} :

\begin{enumerate}[leftmargin=*]
    \item \textbf{Contexte Industriel \& Objectifs} -- que je vais vous présenter maintenant : les enjeux des turbines HP et les problématiques TBC.
    
    \item \textbf{Modélisation Mathématique} -- Achille détaillera la méthode spectrale, les formulations théoriques et l'approche semi-analytique.
    
    \item \textbf{Implémentation Logicielle} -- Lionel présentera l'architecture du code Python, les modules de calcul et l'interface Streamlit.
    
    \item \textbf{Résultats \& Validation} -- Maxime conclura avec les résultats numériques, les études paramétriques et la validation ONERA.
\end{enumerate}~>>
\end{johanbox}

\begin{johanbox}[JOHAN -- Slide 3 : Turbines Haute Pression (1 min 30)]
<<~Commençons par le contexte industriel.

Les turbines haute pression sont des \textbf{composants critiques} des moteurs aéronautiques. Elles fonctionnent dans un \textbf{environnement extrême} :

\begin{itemize}[leftmargin=*]
    \item Températures des gaz de combustion : \textbf{supérieures à 1500°C}
    \item Gradients thermiques \textbf{sévères} : jusqu'à \textbf{200°C sur quelques millimètres}
    \item Contraintes mécaniques intenses : force centrifuge + fatigue cyclique
\end{itemize}

\textbf{Le défi} : l'industrie cherche constamment à augmenter les températures de fonctionnement pour améliorer l'\textbf{efficacité énergétique} des moteurs.

Chaque gain de 10°C représente environ \textbf{1 à 2\%} d'amélioration du rendement thermodynamique.~>>
\end{johanbox}

\begin{johanbox}[JOHAN -- Slide 4 : Architecture TBC Multicouche (1 min 30)]
<<~Pour protéger les aubes de ces conditions extrêmes, on utilise des \textbf{systèmes TBC} -- Thermal Barrier Coating.

\textbf{[Pointer le schéma en coupe des 3 couches]}

Notre système multicouche comprend \textbf{trois couches} :

\textbf{1. La céramique TBC} (en haut) -- 50 à 500 µm
\begin{itemize}[leftmargin=*]
    \item Zircone stabilisée Yttrium (YSZ 7\%)
    \item Assure l'\textbf{isolation thermique principale}
    \item Conductivité très faible : 1{,}5 W/m·K
\end{itemize}

\textbf{2. Le Bond Coat} (couche d'accroche) -- environ 10 µm
\begin{itemize}[leftmargin=*]
    \item Alliage MCrAlY
    \item Assure l'\textbf{adhésion} et la protection contre l'oxydation
\end{itemize}

\textbf{3. Le Substrat} (en bas) -- environ 500 µm
\begin{itemize}[leftmargin=*]
    \item Superalliage base nickel (Inconel 718)
    \item Assure la \textbf{résistance mécanique} structurelle
\end{itemize}~>>
\end{johanbox}

\begin{johanbox}[JOHAN -- Slide 5 : Mécanismes de Défaillance (1 min 30)]
<<~Ces systèmes TBC présentent malheureusement des \textbf{modes de défaillance critiques}.

\textbf{[Pointer les schémas 3D des différents mécanismes]}

\textbf{La délamination} (décohésion) :
\begin{itemize}[leftmargin=*]
    \item Perte d'adhésion à l'interface céramique/bond coat
    \item Due aux contraintes d'arrachement $\sigma_{33}$
\end{itemize}

\textbf{L'écaillage} (spalling) :
\begin{itemize}[leftmargin=*]
    \item Perte de morceaux de revêtement
    \item Réduit drastiquement la protection thermique
\end{itemize}

\textbf{La fissuration} :
\begin{itemize}[leftmargin=*]
    \item Propagation de fissures traversantes
    \item Par concentration de contraintes aux interfaces
\end{itemize}

\textbf{Impact} : Ces phénomènes sont la \textbf{cause principale de défaillance} des systèmes TBC, avec des coûts de maintenance très élevés.~>>
\end{johanbox}

\begin{johanbox}[JOHAN -- Slide 6 : Objectifs du Projet (1 min)]
<<~Face à ces problématiques, notre projet vise à développer un \textbf{outil de prédiction thermomécanique}.

\textbf{4 objectifs principaux} :

\textbf{1. Modéliser} -- la réponse thermomécanique d'architectures multicouches 3D avec une formulation semi-analytique rigoureuse.

\textbf{2. Prédire} -- les zones critiques d'endommagement aux interfaces, identifier les dépassements de seuils de contrainte.

\textbf{3. Quantifier} -- les effets des paramètres clés : anisotropie, contrastes thermiques, épaisseurs.

\textbf{4. Fournir} -- des cartes de sensibilité pour guider la conception et optimiser les aubes pour plus de robustesse.~>>
\end{johanbox}

\begin{johanbox}[JOHAN -- Slide 7 : Partenaires (45 sec)]
<<~Ce projet a été réalisé en partenariat avec trois acteurs majeurs :

\textbf{ONERA} -- Research \& Science
\begin{itemize}[leftmargin=*]
    \item Expertise mécanique des matériaux
    \item Supervision scientifique : Aurélien Vattré
\end{itemize}

\textbf{Safran} -- Industry \& Defense
\begin{itemize}[leftmargin=*]
    \item Contexte industriel et données matériaux
    \item Validation des résultats
\end{itemize}

\textbf{ESTACA} -- Engineering School
\begin{itemize}[leftmargin=*]
    \item Formation ingénieur et encadrement pédagogique
    \item Daniel Gaffié
\end{itemize}~>>
\end{johanbox}

\begin{johanbox}[JOHAN -- Slide 8 : Planning du Projet (45 sec)]
<<~Notre projet s'est déroulé sur \textbf{6 à 8 mois} en 4 phases principales :

\textbf{Phase 1} (Mois 1-2) : Bibliographie \& Théorie
\begin{itemize}[leftmargin=*]
    \item Étude de la méthode spectrale, compréhension des équations
\end{itemize}

\textbf{Phase 2} (Mois 2-4) : Développement Module Calcul
\begin{itemize}[leftmargin=*]
    \item Implémentation Python des solveurs
\end{itemize}

\textbf{Phase 3} (Mois 4-6) : Interface \& Visualisations
\begin{itemize}[leftmargin=*]
    \item Développement Streamlit et Plotly 3D
\end{itemize}

\textbf{Phase 4} (Mois 6-8) : Validation \& Documentation
\begin{itemize}[leftmargin=*]
    \item Tests croisés et rédaction du rapport
\end{itemize}~>>
\end{johanbox}

\begin{johanbox}[JOHAN -- Slide 9 : Méthodologie Générale (1 min)]
<<~Voici le \textbf{flux de calcul} de notre approche semi-analytique.

\textbf{[Pointer le schéma de flux]}

\textbf{Entrées} : Paramètres géométriques ($\alpha$, $L_w$, $\beta$), propriétés matériaux.

\textbf{Thermique} : Résolution par séries de Fourier, équation de conduction.

\textbf{Mécanique} : Construction de la matrice $\Gamma(\tau)$, calcul des valeurs propres.

\textbf{Dommage} : Application des critères D et Tsai-Wu.

\textbf{Sortie} : Profils de contraintes, visualisation 3D, recommandations.

Nous avons également ajouté le logo Safran pour souligner notre partenariat industriel.~>>
\end{johanbox}

\begin{johanbox}[JOHAN -- Slide 10 : Livrables (1 min)]
<<~Notre projet a généré \textbf{trois livrables} principaux :

\textbf{1. Module de Calcul Python} -- environ \textbf{3000 lignes} de code
\begin{itemize}[leftmargin=*]
    \item 5 modules spécialisés dans le répertoire \texttt{core/}
    \item Solveurs thermique et mécanique complets
\end{itemize}

\textbf{2. Interface Streamlit Interactive}
\begin{itemize}[leftmargin=*]
    \item 8 onglets spécialisés
    \item Dashboard avec KPIs en temps réel
    \item Visualisations 3D avec Plotly
\end{itemize}

\textbf{3. Documentation Technique}
\begin{itemize}[leftmargin=*]
    \item Rapport de synthèse complet
    \item Traçabilité Théorie $\leftrightarrow$ Code
    \item Tests et validation
\end{itemize}

Je passe maintenant la parole à Achille pour la modélisation mathématique.~>>
\end{johanbox}

%═══════════════════════════════════════════════════════════════════════════════
\section{PARTIE 2 : Modélisation Mathématique (Achille -- 10 min)}
%═══════════════════════════════════════════════════════════════════════════════

\begin{achillebox}[ACHILLE -- Slide 11 : Introduction Méthode Spectrale (1 min)]
<<~Merci Johan. Je vais maintenant vous présenter les \textbf{fondements mathématiques} de notre approche.

La \textbf{méthode spectrale} repose sur un principe fondamental : la \textbf{décomposition en séries de Fourier}.

\textbf{[Pointer le schéma domaine spatial → domaine spectral]}

\textbf{Principe} : On décompose les champs physiques en séries de Fourier doubles.

La formule de base :
\[
T(x) = \sum T_{mn} \sin(\delta x)
\]

\textbf{Avantages vs Éléments Finis (FEM)} :
\begin{itemize}[leftmargin=*]
    \item \textbf{Semi-Analytique} : pas de discrétisation spatiale
    \item \textbf{Rapidité} : résolution directe sans maillage
    \item \textbf{Précision} : erreurs numériques minimales
\end{itemize}~>>
\end{achillebox}

\begin{achillebox}[ACHILLE -- Slide 12 : Représentation Spectrale de la Température (1 min)]
<<~La \textbf{représentation spectrale} de la température est la première étape clé.

Le champ de température s'écrit :
\[
T(x_1, x_2, x_3) = \sum T_{mn}(x_3) \sin(\delta_1 x_1) \sin(\delta_2 x_2)
\]

avec le nombre d'onde :
\[
\delta = \frac{\pi}{L_w}
\]

\textbf{[Pointer la visualisation 3D des ondes]}

\textbf{$L_w$} représente la \textbf{longueur d'onde} de la perturbation thermique latérale.

\textbf{$T_{mn}(x_3)$} sont les amplitudes modales -- elles dépendent \textbf{uniquement de la profondeur}.

\textbf{Transformation} : On passe d'équations 3D complexes à des équations 1D plus simples à résoudre. C'est une \textbf{simplification majeure}.~>>
\end{achillebox}

\begin{achillebox}[ACHILLE -- Slide 13 : Solution Thermique par Couche (1 min)]
<<~\textbf{Étape 3} : Dans chaque couche $i$, la solution de l'équation de conduction est \textbf{exponentielle}.

\[
T^{(i)}(x_3) = A^{(i)} e^{\lambda^{(i)} x_3} + B^{(i)} e^{-\lambda^{(i)} x_3}
\]

L'exposant thermique $\lambda$ dépend du ratio de conductivités :
\[
\lambda^{(i)} = \delta_\eta \sqrt{\frac{k_{\eta\eta}^{(i)}}{k_{33}^{(i)}}}
\]

\textbf{Solution Exponentielle} : Valide pour chaque couche $i$.

\textbf{Anisotropie} : $\lambda$ dépend du ratio $k_{\eta}/k_{33}$.

\textbf{Système Global} : 2N inconnues (A, B) déterminées par continuité aux interfaces.~>>
\end{achillebox}

\begin{achillebox}[ACHILLE -- Slide 14 : Loi de Hooke Thermo-Élastique (1 min)]
<<~\textbf{Étape 4} : La formulation mécanique repose sur la \textbf{loi de Hooke thermo-élastique}.

\[
\sigma_{ij} = C_{ijkl} \left( \varepsilon_{kl} - \alpha_{kl} \Delta T \right)
\]

\textbf{[Pointer le schéma tensoriel avec la matrice C]}

Le tenseur de rigidité $C_{ijkl}$ contient les propriétés élastiques de chaque couche.

\textbf{Propriétés par couche} :
\begin{center}
\begin{tabular}{lcc}
\textbf{Matériau} & $C_{11}$ & $\alpha$ \\
\hline
Substrat (Ni) & 260 GPa & 13 $\mu\varepsilon$/K \\
Bond Coat & 180 GPa & 14 $\mu\varepsilon$/K \\
Céramique & 50 GPa & 10 $\mu\varepsilon$/K \\
\end{tabular}
\end{center}

Le \textbf{contraste important} entre les propriétés ($C_{11}$ varie de 260 à 50 GPa) est la source principale des contraintes aux interfaces.~>>
\end{achillebox}

\begin{achillebox}[ACHILLE -- Slide 15 : Ansatz de Déplacement Modal (1 min)]
<<~\textbf{Étape 5} : On suppose une \textbf{forme particulière} pour les déplacements.

\textbf{Déplacement} :
\[
u_i = V_i(x_3) \times \text{trig}(\delta x_1, \delta x_2)
\]

\textbf{Amplitude} :
\[
V_i(x_3) = A_i \exp(\tau x_3)
\]

où $\tau$ est la \textbf{valeur propre} à déterminer.

\textbf{[Pointer la visualisation 3D des modes propres]}

\textbf{Concept clé : Condition de Radiation}
\[
\boxed{\text{Re}(\tau) < 0}
\]

Cette condition assure la \textbf{stabilité} et la \textbf{décroissance} des perturbations en profondeur.~>>
\end{achillebox}

\begin{achillebox}[ACHILLE -- Slide 16 : Matrice Dynamique $\Gamma(\tau)$ (1 min 30) -- TECHNIQUE]
<<~\textbf{Étape 6} : Le cœur de notre méthode est la \textbf{matrice dynamique} $\Gamma(\tau)$.

\textbf{[Pointer la matrice affichée]}

\[
\Gamma(\tau) = \begin{pmatrix}
L_{11} & L_{12} & L_{13} \\
L_{21} & L_{22} & L_{23} \\
\textcolor{red}{-L_{31}} & \textcolor{red}{-L_{32}} & L_{33}
\end{pmatrix}
\]

\textbf{Structure} : Assemblage des 9 opérateurs $L_{jk}$.

\textbf{Diagonale} : Résistance à la déformation ($L_{ii}$).

\textbf{Point CRITIQUE} : \textbf{Antisymétrie} des termes croisés :
\[
\boxed{L_{13} = -L_{31}}
\]

Cette propriété provient de l'équilibre en direction $x_3$ et est \textbf{essentielle} pour la physique correcte.

L'équation caractéristique est :
\[
\det(\Gamma(\tau)) = 0
\]~>>
\end{achillebox}

\begin{achillebox}[ACHILLE -- Slide 17 : Résolution Équation Caractéristique (1 min)]
<<~La résolution de $\det(\Gamma(\tau)) = 0$ donne un \textbf{polynôme d'ordre 6}.

\textbf{Astuce mathématique} : Changement de variable $X = \tau^2$.

\textbf{[Pointer le schéma de réduction]}

\textbf{Input} : Polynôme ordre 6 → $\det(\Gamma(\tau)) = 0$

\textbf{Intermédiaire} : Polynôme cubique → $c_6 X^3 + c_4 X^2 + c_2 X + c_0 = 0$

\textbf{Output} : \textbf{3 racines sélectionnées}

\textbf{Sélection Physique} : Dans le plan complexe, on ne conserve que les ondes \textbf{évanescentes} avec $\text{Re}(\tau) < 0$.

\textbf{[Pointer le graphique du plan complexe avec racines vertes/rouges]}

Les racines rouges (Re > 0) sont rejetées car non physiques.~>>
\end{achillebox}

\begin{achillebox}[ACHILLE -- Slide 18 : Assemblage 9×9 avec Sollicitation Thermique (1 min 30)]
<<~\textbf{Étape 7} : On assemble un système linéaire global $9 \times 9$.

\textbf{[Pointer la visualisation de la matrice bloc-diagonale]}

\[
[K_{Dyn}] \cdot \{A\} = \{F_{Th}\}
\]

La matrice $K_{Dyn}$ est \textbf{bloc-diagonale} :
\begin{itemize}[leftmargin=*]
    \item 3 blocs $\Gamma(\tau_r)$ sur la diagonale
    \item Chaque bloc correspond à une valeur propre
\end{itemize}

Le vecteur $F_{Th}$ contient les \textbf{termes sources thermiques} $Q_\alpha$ :
\begin{itemize}[leftmargin=*]
    \item Couplage entre dilatation thermique et rigidités
\end{itemize}

\textbf{Résolution} : Inversion matricielle pour trouver les 9 amplitudes $A$.~>>
\end{achillebox}

\begin{achillebox}[ACHILLE -- Slide 19 : Assemblage Multicouche (1 min)]
<<~\textbf{Étape 8} : Pour $N$ couches, on assemble un système global.

\textbf{[Pointer le diagramme multicouche]}

\textbf{Conditions aux limites} : Surface libre en haut et en bas ($\sigma \cdot n = 0$).

\textbf{Conditions de continuité} : Aux interfaces, continuité de $u$ et $\sigma$.

\textbf{Tableau de réduction} :
\begin{center}
\begin{tabular}{lcc}
 & N = 3 couches & Spectral \\
\hline
Théorique & 27 équations & \\
Spectral & & \textcolor{red}{\textbf{18 équations}} \\
\end{tabular}
\end{center}

\textbf{Gain : -9 équations} ! L'équilibre volumique est \textbf{implicitement satisfait} par l'approche modale.

Je passe la parole à Lionel pour l'implémentation logicielle.~>>
\end{achillebox}

%═══════════════════════════════════════════════════════════════════════════════
\section{PARTIE 3 : Implémentation Logicielle (Lionel -- 10 min)}
%═══════════════════════════════════════════════════════════════════════════════

\begin{lionelbox}[LIONEL -- Slide 20 : Critères d'Endommagement (1 min 30)]
<<~Merci Achille. Avant de présenter l'implémentation, je vais d'abord expliquer les \textbf{critères d'endommagement} que nous calculons.

\textbf{Indicateur de Dommage D} :
\[
D = \max \left( \frac{|\sigma_{ij}|}{\sigma_{crit}^{ij}} \right)
\]

\textbf{[Pointer la jauge colorée]}

\textbf{Interprétation} :
\begin{itemize}[leftmargin=*]
    \item $D < 0.5$ : Zone \textbf{Sûre} (vert)
    \item $0.5 \leq D < 0.8$ : Zone de \textbf{Prudence} (orange)
    \item $D \geq 0.8$ : Zone \textbf{Critique/Rupture} (rouge)
\end{itemize}

\textbf{Critère de Tsai-Wu} (matériaux anisotropes) :
\[
F = F_i \sigma_i + F_{ij} \sigma_i \sigma_j \geq 1
\]

\textbf{Contraintes critiques} : Substrat 1000 MPa, Bond Coat 500 MPa, Céramique 150 MPa.~>>
\end{lionelbox}

\begin{lionelbox}[LIONEL -- Slide 21 : Stabilité Numérique (1 min 30)]
<<~Un défi majeur de notre projet : la \textbf{stabilité numérique}.

\textbf{Problème} : Le conditionnement de la matrice peut dépasser $10^{30}$ !

\textbf{[Pointer le schéma : Chaos → Régularisation → Stabilité]}

\textbf{Nos solutions} :

\textbf{1. Préconditionnement (Scaling)}
\[
K' = D_r \cdot K \cdot D_c
\]
Normalisation des ordres de grandeur.

\textbf{2. Régularisation Tikhonov (SVD)}
\[
K = U \Sigma V^T
\]
Filtrage des valeurs singulières : $\sigma_i \to 0$ si $\sigma_i < \epsilon$.

\textbf{3. Normalisation Globale}
\[
C_{ref} = 200 \text{ GPa}
\]

Ces techniques nous permettent d'atteindre une précision de $10^{-8}$ sur les résultats.~>>
\end{lionelbox}

\begin{lionelbox}[LIONEL -- Slide 22 : Architecture Logicielle (1 min 30)]
<<~Voici l'\textbf{architecture de notre code} Python.

\textbf{[Pointer le schéma de l'architecture]}

\textbf{Répertoire} \texttt{core/} -- 5 modules :

\begin{itemize}[leftmargin=*]
    \item \texttt{mechanical\_pdf.py} : Solveur Spectral (\textbf{1031 lignes})
    \item \texttt{mechanical.py} : Assemblage Multicouche (\textbf{1507 lignes})
    \item \texttt{damage\_analysis.py} : Critères Endommagement (\textbf{367 lignes})
    \item \texttt{calculation.py} : Solveur Thermique (\textbf{269 lignes})
    \item \texttt{constants.py} : Données Matériaux
\end{itemize}

\textbf{Volume total} : environ \textbf{3000 lignes} de code.

\textbf{Langage} : Python 3.x

\textbf{Organisation} : Modulaire et orientée objet pour faciliter la maintenance.~>>
\end{lionelbox}

\begin{lionelbox}[LIONEL -- Slide 23 : Interface Streamlit - Dashboard (2 min)]
<<~Notre interface utilisateur est construite avec \textbf{Streamlit}.

\textbf{[Montrer la capture d'écran du dashboard]}

\textbf{6 KPI Cards} : Indicateurs clés en temps réel
\begin{itemize}[leftmargin=*]
    \item Température interface : 1050°C (seuil critique)
    \item Épaisseur TBC : $320 \pm 20$ µm
    \item Indicateur D : 0.75 (performance acceptable)
    \item Temps de réponse : 12 ms (optimal)
    \item Connectivité : Data stream 5G stable
    \item Santé système : 98\%, maintenance dans 30j
\end{itemize}

\textbf{Badge Conforme ONERA} : Validation automatique.

\textbf{Visualisation 3D} : Cartographie thermique interactive (Plotly).

\textbf{Aide à la décision} : Jauge de risque + Radar multi-critères.~>>
\end{lionelbox}

\begin{lionelbox}[LIONEL -- Slide 24 : Résultats Thermiques (1 min 30)]
<<~Passons aux \textbf{premiers résultats} : le profil thermique.

\textbf{[Pointer le graphique T(z)]}

\textbf{Observations clés} :

\textbf{Gradient thermique} par couche :
\begin{itemize}[leftmargin=*]
    \item Substrat : gradient \textbf{faible} ($\sim$20°C/mm)
    \item Céramique : gradient \textbf{extrême} ($\sim$400°C/mm)
\end{itemize}

\textbf{Température à l'interface} : environ \textbf{1050°C}

Comparée au seuil critique $T_{crit} = 1100$°C : nous avons une \textbf{marge de 50°C}.

\textbf{Influence des paramètres} :
\begin{itemize}[leftmargin=*]
    \item Épaisseur TBC $\uparrow$ $\Rightarrow$ $T_{interface}$ $\downarrow$
    \item Conductivité $\downarrow$ $\Rightarrow$ Isolation $\uparrow$
\end{itemize}

Je laisse maintenant la parole à Maxime pour les résultats mécaniques et la conclusion.~>>
\end{lionelbox}

%═══════════════════════════════════════════════════════════════════════════════
\section{PARTIE 4 : Résultats, Validation et Conclusion (Maxime -- 10 min)}
%═══════════════════════════════════════════════════════════════════════════════

\begin{maximebox}[MAXIME -- Slide 25 : Résultats Mécaniques - Contraintes (1 min 30)]
<<~Merci Lionel. Je vais maintenant vous présenter les \textbf{résultats mécaniques}.

\textbf{[Pointer les deux graphiques de profils $\sigma(z)$]}

\textbf{Graph 1 : Contrainte d'arrachement $\sigma_{33}$}
\begin{itemize}[leftmargin=*]
    \item Pic maximal à l'interface BC/Céramique
    \item Valeur : environ \textbf{200 MPa}
    \item Seuil critique : 150 MPa → \textbf{Dépassé !}
\end{itemize}

\textbf{Graph 2 : Contrainte de cisaillement $\sigma_{13}$}
\begin{itemize}[leftmargin=*]
    \item Pic à \textbf{250 MPa} à l'interface
    \item Décroissance rapide dans le substrat
\end{itemize}

\textbf{Points clés} :
\begin{itemize}[leftmargin=*]
    \item Risque de délamination si $\sigma_{33} > \sigma_{crit}$
    \item Valeurs typiques : 50 à 200 MPa
\end{itemize}~>>
\end{maximebox}

\begin{maximebox}[MAXIME -- Slide 26 : Études Paramétriques (1 min 30)]
<<~Nous avons réalisé des \textbf{études de sensibilité} pour identifier les leviers de conception.

\textbf{[Pointer les courbes de tendance et la heatmap]}

\textbf{Influence de l'épaisseur ($\alpha$)} :
\begin{itemize}[leftmargin=*]
    \item $\alpha \uparrow$ $\Rightarrow$ D $\downarrow$ (bénéfique)
    \item Un TBC plus épais réduit les contraintes
\end{itemize}

\textbf{Influence de la longueur d'onde ($L_w$)} :
\begin{itemize}[leftmargin=*]
    \item $L_w \downarrow$ $\Rightarrow$ D $\uparrow$ (critique)
    \item Perturbations courtes = gradients plus forts
\end{itemize}

\textbf{Influence de la conductivité ($\beta$)} :
\begin{itemize}[leftmargin=*]
    \item $\beta \downarrow$ $\Rightarrow$ D $\uparrow$
\end{itemize}

\textbf{Recommandation} : $\alpha \geq 0.2$ pour applications haute pression.

\textbf{Physique} : TBC épais = meilleure isolation et gradients réduits.~>>
\end{maximebox}

\begin{maximebox}[MAXIME -- Slide 27 : Validation ONERA/Safran (1 min 30)]
<<~Nos résultats ont été \textbf{validés} par comparaison aux données de référence ONERA/Safran.

\textbf{[Pointer le tableau de comparaison et le badge]}

\textbf{Comparaison Code vs Référence} :
\begin{center}
\begin{tabular}{lccc}
 & Valeur Code & Valeur Réf & Écart \\
\hline
$C_{11}$ Substrat & 145 GPa & 144.8 GPa & \textbf{0.15\%} \\
$C_{12}$ Substrat & 60 GPa & 60.0 GPa & \textbf{0.0\%} \\
$C_{11}$ Céramique & 420 GPa & 418.5 GPa & \textbf{0.36\%} \\
\end{tabular}
\end{center}

\textbf{Écarts < 1\%} sur les modules élastiques.

\textbf{Contraintes} : Résultat code = 600 MPa, dans la plage FEM ONERA (400--800 MPa).

$\Rightarrow$ Badge \textbf{``CONFORME ONERA''} affiché dans le dashboard.~>>
\end{maximebox}

\begin{maximebox}[MAXIME -- Slide 28 : Démonstration Application (1 min)]
<<~Voici le \textbf{parcours utilisateur} de notre application.

\textbf{[Pointer les 4 étapes du workflow]}

\textbf{1. Paramétrage}
\begin{itemize}[leftmargin=*]
    \item Entrée des paramètres : $\alpha$, $\beta$, $L_w$
\end{itemize}

\textbf{2. Calcul Rapide}
\begin{itemize}[leftmargin=*]
    \item Résolution spectrale en \textbf{moins d'une seconde}
\end{itemize}

\textbf{3. Visualisation}
\begin{itemize}[leftmargin=*]
    \item Dashboard et profils 3D interactifs
\end{itemize}

\textbf{4. Rapport}
\begin{itemize}[leftmargin=*]
    \item Export PDF/CSV avec conseils automatiques
\end{itemize}

\textbf{Interface} : Streamlit interactive

\textbf{Feedback} : Instantané

\textbf{Usage} : Aide à la décision rapide.~>>
\end{maximebox}

\begin{maximebox}[MAXIME -- Slide 29 : Limites du Modèle (1 min)]
<<~Notre modèle présente certaines \textbf{limites} à mentionner.

\textbf{[Pointer le schéma Modèle vs Réalité]}

\textbf{Géométrie} :
\begin{itemize}[leftmargin=*]
    \item Modèle : plaque plane
    \item Réalité : courbure complexe d'aube
\end{itemize}

\textbf{Thermique} :
\begin{itemize}[leftmargin=*]
    \item Modèle : stationnaire
    \item Réalité : cycles thermiques (start/stop)
\end{itemize}

\textbf{Mécanique} :
\begin{itemize}[leftmargin=*]
    \item Modèle : élastique linéaire
    \item Réalité : fluage, plasticité possibles
\end{itemize}

\textbf{Validité} : $T < 1100$°C, $\sigma < 500$ MPa

Ces limites sont appropriées pour un \textbf{pré-dimensionnement} avant analyse FEM détaillée.~>>
\end{maximebox}

\begin{maximebox}[MAXIME -- Slide 30 : Perspectives d'Amélioration (1 min)]
<<~Pour la suite, plusieurs \textbf{pistes d'évolution} s'offrent au projet.

\textbf{[Pointer le schéma circulaire des perspectives]}

\textbf{Extension Géométrique}
\begin{itemize}[leftmargin=*]
    \item Géométrie cylindrique et complexe
\end{itemize}

\textbf{Fatigue Thermique}
\begin{itemize}[leftmargin=*]
    \item Cyclage et durée de vie
\end{itemize}

\textbf{Machine Learning}
\begin{itemize}[leftmargin=*]
    \item Surrogate model pour optimisation rapide
\end{itemize}

\textbf{Matériaux Avancés}
\begin{itemize}[leftmargin=*]
    \item Base de données étendue et CMC
\end{itemize}

Ces développements permettraient d'atteindre un outil de conception industriel complet.~>>
\end{maximebox}

\begin{maximebox}[MAXIME -- Slide 31 : Conclusion Générale (1 min)]
<<~En conclusion, notre projet a atteint ses \textbf{4 objectifs} principaux.

\textbf{OBJECTIFS ATTEINTS} $\checkmark$
\begin{itemize}[leftmargin=*]
    \item Modélisation spectrale fonctionnelle
    \item Prédiction des zones critiques aux interfaces
\end{itemize}

\textbf{OUTIL VALIDÉ} $\checkmark$
\begin{itemize}[leftmargin=*]
    \item Python/Streamlit opérationnel
    \item Validation ONERA conforme
\end{itemize}

\textbf{TRAÇABILITÉ} $\checkmark$
\begin{itemize}[leftmargin=*]
    \item Correspondance stricte Théorie $\leftrightarrow$ Code
\end{itemize}

\textbf{POTENTIEL INDUSTRIEL} $\checkmark$
\begin{itemize}[leftmargin=*]
    \item Aide à la conception et optimisation rapide
\end{itemize}~>>
\end{maximebox}

\begin{maximebox}[MAXIME -- Slide 32 : Remerciements (30 sec)]
<<~Nous tenons à remercier sincèrement :

\textbf{ONERA} : Aurélien Vattré pour la supervision scientifique.

\textbf{ESTACA} : Daniel Gaffié pour l'encadrement pédagogique.

\textbf{Autres} : Toutes les sources de recherches consultées.

\textbf{Équipe Projet 5A-IDSA} :
\begin{center}
Johan GUITTON · Lionel DAURIAC · Achille VOISIN · Maxime LEBOEUF
\end{center}

\vspace{0.5cm}
\begin{center}
\textbf{\Large MERCI DE VOTRE ATTENTION}\\[0.5cm]
\large Questions ?
\end{center}~>>
\end{maximebox}

%═══════════════════════════════════════════════════════════════════════════════
\section*{BACKUP : Réponses aux Questions Anticipées}
\addcontentsline{toc}{section}{BACKUP : Questions Anticipées}
%═══════════════════════════════════════════════════════════════════════════════

\begin{tcolorbox}[colback=codecolor!5, colframe=codecolor, title={\textbf{Q : Pourquoi méthode spectrale vs éléments finis ?}}]
La méthode spectrale offre :
\begin{itemize}[leftmargin=*]
    \item Pas de maillage spatial → pas de problèmes de convergence
    \item Résolution quasi-instantanée → idéal pour études paramétriques
    \item Précision analytique contrôlée par le nombre de modes
    \item Complémentarité avec FEM pour pré-dimensionnement
\end{itemize}
\end{tcolorbox}

\begin{tcolorbox}[colback=codecolor!5, colframe=codecolor, title={\textbf{Q : Comment gérez-vous l'antisymétrie $L_{13} = -L_{31}$ ?}}]
Dans le code \texttt{mechanical\_pdf.py}, nous implémentons explicitement :

\texttt{M[2,0] = -K13\_coeff * tau}

Le signe négatif est vérifié par des tests unitaires qui vérifient $\Gamma_{13} + \Gamma_{31} = 0$.
\end{tcolorbox}

\begin{tcolorbox}[colback=codecolor!5, colframe=codecolor, title={\textbf{Q : Quelle précision numérique ?}}]
Nous atteignons $10^{-6}$ à $10^{-8}$ grâce à :
\begin{itemize}[leftmargin=*]
    \item Régularisation SVD filtrant les modes instables
    \item Préconditionnement scaling : $\kappa(K)$ de $10^{30}$ à $10^{6}$
    \item Normalisation $C_{ref} = 200$ GPa
\end{itemize}
Résidu typique : $< 10^{-10}$.
\end{tcolorbox}

\vspace{1cm}
\begin{center}
\rule{0.8\textwidth}{0.4pt}\\[0.5cm]
\textit{Fin du script de soutenance -- Version 2}\\
\textbf{32 slides · 40 minutes + 15 min questions}
\end{center}

\end{document}
