\documentclass[a4paper,11pt]{article}
\usepackage[utf8]{inputenc}
\usepackage[T1]{fontenc}
\usepackage[french]{babel}
\usepackage{amsmath, amssymb, amsfonts}
\usepackage{graphicx}
\usepackage{hyperref}
\usepackage{geometry}
\usepackage{xcolor}
\usepackage{tikz}
\usetikzlibrary{arrows.meta, positioning, shapes.geometric, fit, calc}
\usepackage{float}
\usepackage{booktabs}
\usepackage{tcolorbox}
\tcbuselibrary{skins,breakable}
\usepackage{fancyhdr}
\usepackage{array}
\usepackage{longtable}

\geometry{hmargin=2cm,vmargin=2cm}
\setlength{\headheight}{14pt}

% Couleurs
\definecolor{theoremcolor}{RGB}{59, 130, 246}
\definecolor{definitioncolor}{RGB}{16, 185, 129}
\definecolor{warningcolor}{RGB}{239, 68, 68}
\definecolor{codecolor}{RGB}{30, 41, 59}

% Boîtes colorées
\newtcolorbox{theorembox}[1][]{
    colback=theoremcolor!5,
    colframe=theoremcolor,
    fonttitle=\bfseries,
    title={#1},
    breakable
}
\newtcolorbox{definitionbox}[1][]{
    colback=definitioncolor!5,
    colframe=definitioncolor,
    fonttitle=\bfseries,
    title={#1},
    breakable
}
\newtcolorbox{warningbox}[1][]{
    colback=warningcolor!5,
    colframe=warningcolor,
    fonttitle=\bfseries,
    title={#1},
    breakable
}
\newtcolorbox{codebox}[1][]{
    colback=codecolor!5,
    colframe=codecolor,
    fonttitle=\bfseries\ttfamily,
    title={#1},
    breakable
}

% En-tête
\pagestyle{fancy}
\fancyhf{}
\rhead{Projet IDSA 5A - TBC}
\lhead{Rapport de Synthèse}
\rfoot{Page \thepage}

\title{\textbf{Rapport de Synthèse}\\[0.5cm] 
\Large Modélisation Thermo-Mécanique des Systèmes TBC\\[0.3cm]
\normalsize Méthode Spectrale et Implémentation Python/Streamlit\\[1cm]
\small Projet Industriel 5A-IDSA - Encadrement ONERA (A. Vattré)}
\author{Documentation Technique}
\date{\today}

\begin{document}

% Page de titre
\begin{titlepage}
\begin{center}
\vspace*{1cm}

\begin{tcolorbox}[colback=theoremcolor!10, colframe=theoremcolor, width=\textwidth]
\centering
\Huge\textbf{Rapport de Synthèse}\\[0.5cm]
\Large Modélisation Thermo-Mécanique\\des Systèmes Barrière Thermique (TBC)
\end{tcolorbox}

\vspace{1cm}
\large Méthode Spectrale et Implémentation Python/Streamlit\\[0.3cm]
\normalsize Projet Industriel 5A-IDSA

\vspace{2cm}

\begin{tikzpicture}[scale=0.8]
    % Simplified TBC layer visualization
    \fill[gray!40] (-3,0) rectangle (3,1.5);
    \fill[orange!40] (-3,1.5) rectangle (3,1.8);
    \fill[blue!30] (-3,1.8) rectangle (3,3);
    \draw[thick] (-3,0) rectangle (3,3);
    \draw[dashed] (-3,1.5) -- (3,1.5);
    \draw[dashed] (-3,1.8) -- (3,1.8);
    \node at (0,0.75) {Substrat};
    \node at (0,1.65) {\footnotesize BC};
    \node at (0,2.4) {TBC};
    \draw[red, ->, very thick] (3.5,3) -- (4.5,3) node[right] {$T_{hot}$};
    \draw[blue, ->, very thick] (3.5,0) -- (4.5,0) node[right] {$T_{cold}$};
\end{tikzpicture}

\vspace{2cm}

\begin{tabular}{rl}
\textbf{Projet :} & Industriel 5A-IDSA \\
\textbf{Encadrement :} & ONERA (A. Vattré) \\
\textbf{Partenaires :} & ESTACA / Safran \\
\textbf{Date :} & \today
\end{tabular}

\vfill

\begin{tcolorbox}[colback=definitioncolor!10, colframe=definitioncolor, width=0.8\textwidth]
\centering\small
\textbf{Mots-clés :} Barrière thermique, Méthode spectrale, Multicouche, \\
Analyse modale, Critères d'endommagement, Streamlit
\end{tcolorbox}

\end{center}
\end{titlepage}

\tableofcontents
\newpage

%==============================================================================
\section*{Résumé Exécutif}
\addcontentsline{toc}{section}{Résumé Exécutif}
%==============================================================================

\begin{tcolorbox}[colback=theoremcolor!5, colframe=theoremcolor, title=Projet 5A-IDSA : Évaluation Thermomécanique des Aubes de Turbines]
\textbf{Encadrement :} ONERA (Aurélien Vattré) / ESTACA (Daniel Gaffié)

\textbf{Objectif :} Développer un outil de simulation numérique pour l'évaluation thermomécanique des zones critiques d'endommagement dans les aubes de turbines multicouches nouvelle génération.
\end{tcolorbox}

\textbf{Contexte Industriel :}
Les aubes de turbines sont soumises à des conditions extrêmes : températures dépassant 1000°C, gradients thermiques sévères (jusqu'à 200°C sur quelques millimètres), et chargements mécaniques cycliques. Des architectures multicouches combinant substrat métallique, couche d'accroche et revêtement céramique sont utilisées pour améliorer leur durabilité.

\textbf{Méthodologie Développée :}
\begin{itemize}
    \item Approche \textbf{semi-analytique} basée sur la décomposition spectrale (séries de Fourier doubles)
    \item Construction de la matrice dynamique $\Gamma(\tau)$ avec opérateurs $L_{jk}$
    \item Résolution du problème aux valeurs propres $\det(\Gamma(\tau)) = 0$
    \item Assemblage matriciel bloc-diagonal $9 \times 9$ avec sollicitation thermique
    \item Critères d'endommagement D et Tsai-Wu pour identification des zones critiques
\end{itemize}

\textbf{Résultats Clés :}
\begin{center}
\begin{tabular}{lcc}
\toprule
\textbf{Métrique} & \textbf{Plage Typique} & \textbf{Seuil Critique} \\
\midrule
Température interface substrat/bondcoat & 800--1050°C & $T_{crit} = 1100$°C \\
Indicateur d'endommagement D & 0.2--0.8 & $D \geq 1$ (rupture) \\
Contrainte normale $\sigma_{33}$ max & 50--200 MPa & $\sigma_t^{ceramic} = 150$ MPa \\
Contrainte cisaillement $\sigma_{13}$ max & 20--80 MPa & $\tau_{crit} = 120$ MPa \\
\bottomrule
\end{tabular}
\end{center}

\textbf{Livrables du Projet :}
\begin{enumerate}
    \item \textbf{Module de calcul} : \texttt{core/mechanical\_pdf.py}
    \item \textbf{Interface Streamlit} : Application interactive avec dashboard et visualisations 3D
    \item \textbf{Documentation technique} : Ce rapport de synthèse
\end{enumerate}

\newpage

%==============================================================================
\section{Introduction}
%==============================================================================

\subsection{Contexte Industriel : Aubes de Turbines Nouvelle Génération}

Les aubes de turbines constituent l'un des composants les plus critiques des moteurs aéronautiques. Situées en aval de la chambre de combustion, elles sont exposées à des gaz à très haute température (supérieure à 1500°C) tout en devant supporter des contraintes mécaniques élevées dues à la rotation.

\begin{warningbox}[Enjeux de Durabilité]
Les phénomènes d'endommagement aux interfaces entre couches représentent la principale cause de défaillance des systèmes TBC :
\begin{itemize}
    \item \textbf{Délamination} : Décohésion entre la couche céramique et le substrat
    \item \textbf{Écaillage} : Perte de morceaux de revêtement céramique
    \item \textbf{Fissuration} : Propagation de fissures dans les zones de concentration de contraintes
\end{itemize}
\end{warningbox}

\subsection{Objectifs du Projet}

Ce projet vise à :
\begin{enumerate}
    \item \textbf{Modéliser} la réponse thermomécanique d'architectures multicouches à partir d'un modèle tridimensionnel simplifié (méthode spectrale)
    \item \textbf{Prédire} les zones critiques d'endommagement situées aux interfaces
    \item \textbf{Quantifier} les effets de l'anisotropie élastique, des contrastes thermiques, et de l'épaisseur des couches
    \item \textbf{Fournir} une carte de sensibilité pour guider la conception de futures aubes plus robustes
\end{enumerate}

%==============================================================================
\section{Système Multicouche Étudié}
%==============================================================================

Le système TBC (Thermal Barrier Coating) comprend $N$ couches empilées selon la direction normale $x_3$ :

\begin{center}
\begin{tikzpicture}[scale=1.1, every node/.style={font=\small}]
    % Substrat
    \fill[gray!40] (0,0) rectangle (8,1.8);
    \node at (4,0.9) {\textbf{Couche 1 : Substrat (Superalliage Ni)}};
    \node[right] at (8.1,1.4) {$h_1 = 500$ µm};
    
    % Bond Coat
    \fill[orange!50] (0,1.8) rectangle (8,2.1);
    \node at (4,1.95) {\footnotesize Couche 2 : BondCoat (MCrAlY)};
    \node[right] at (8.1,1.95) {\footnotesize $h_2 = 10$ µm};
    
    % TBC Céramique
    \fill[blue!30] (0,2.1) rectangle (8,3.5);
    \node at (4,2.8) {\textbf{Couche 3 : Céramique (YSZ)}};
    \node[right] at (8.1,3.0) {$h_3 = \alpha \cdot h_1$};
    
    % Axes
    \draw[->] (-0.5,0) -- (-0.5,4) node[above] {$x_3$};
    \draw[->] (0,-0.3) -- (8.5,-0.3) node[right] {$x_1, x_2$};
    
    % Températures
    \draw[red, thick, ->, >=stealth] (8.3,3.5) -- (9.3,3.5) node[right] {$T_{top} \approx 1400$°C};
    \draw[blue, thick, ->, >=stealth] (8.3,0) -- (9.3,0) node[right] {$T_{bot} \approx 500$°C};
    
    % Interfaces
    \draw[dashed, thick] (-0.2,1.8) -- (8.2,1.8);
    \draw[dashed, thick] (-0.2,2.1) -- (8.2,2.1);
    
    % Épaisseur totale
    \draw[<->] (-1,0) -- (-1,3.5) node[midway, left] {$H = \sum h_i$};
    
    % Zones critiques
    \node[red, font=\scriptsize] at (4,1.8) [below right] {$\leftarrow$ Interface critique};
\end{tikzpicture}
\end{center}

\subsection{Paramètres Adimensionnels Clés}

Les paramètres d'entrée de l'application sont :
\begin{align}
    \alpha &= \frac{h_3}{h_1} \quad \text{(ratio épaisseur céramique/substrat, typiquement 0.1 à 1.0)} \\
    \beta &= \frac{k_3}{k_1} \quad \text{(ratio conductivité thermique)} \\
    L_w &\quad \text{(longueur d'onde de la perturbation latérale, en mètres)} \\
    \delta_1 = \delta_2 &= \frac{\pi}{L_w} \quad \text{(nombres d'onde spectraux)}
\end{align}

\subsection{Propriétés Matériaux}

Les propriétés élastiques et thermiques de chaque couche sont définies dans le module \texttt{core/constants.py}. Les valeurs sont issues des publications ONERA/Safran (voir Section~\ref{sec:validation}).

\begin{center}
\begin{tabular}{lccc}
\toprule
\textbf{Propriété} & \textbf{Substrat (Ni)} & \textbf{BondCoat} & \textbf{Céramique (YSZ)} \\
\midrule
$C_{11}$ (GPa) & 259.6 & 180 & 50 \\
$C_{12}$ (GPa) & 179.0 & 120 & 10 \\
$C_{44}$ (GPa) & 109.6 & 80 & 20 \\
$\alpha$ (K$^{-1}$) & $13 \times 10^{-6}$ & $14 \times 10^{-6}$ & $10 \times 10^{-6}$ \\
\bottomrule
\end{tabular}
\end{center}

%==============================================================================
\section{Modélisation Mathématique}
%==============================================================================

Cette section présente la méthodologie complète de résolution, étape par étape.

\subsection{Étape 1-2 : Représentation Spectrale de la Température}

\begin{theorembox}[Développement en Séries de Fourier]
La température est développée en série double de Fourier :
\begin{equation}
\boxed{
    T(x_\alpha, x_3) = \sum_{m_\alpha, m_\beta=1}^{\infty} T_{m_\alpha m_\beta}(x_3) \sin(\delta_\alpha x_\alpha) \sin(\delta_\beta x_\beta)
}
\end{equation}
avec les nombres d'onde $\delta_\alpha = \frac{m_\alpha \pi}{L_\alpha}$.
\end{theorembox}

\subsection{Étape 3 : Solution Thermique par Couche}

Dans chaque couche $i$, la solution de l'équation de conduction est :
\begin{equation}
    T^{(i)}(x_3) = A^{(i)} e^{\lambda^{(i)} x_3} + B^{(i)} e^{-\lambda^{(i)} x_3}
\end{equation}
avec l'exposant thermique :
\begin{equation}
    \lambda^{(i)} = \delta_\eta \sqrt{\frac{k_{\eta\eta}^{(i)}}{k_{33}^{(i)}}}
\end{equation}

L'implémentation de cette étape se trouve dans le module \texttt{core/calculation.py}, fonction \texttt{solve\_tbc\_model\_v2()}.

\subsection{Étape 4 : Loi de Comportement Thermo-Élastique}

\begin{definitionbox}[Loi de Hooke avec Effet Thermique]
\begin{equation}
    \sigma_{ij}(x) = C_{ijkl}(x_3) \left(\varepsilon_{kl}(x) - \alpha_{kl}(x_3) T(x)\right)
\end{equation}
où $C_{ijkl}$ est le tenseur de rigidité et $\alpha_{kl}$ les coefficients de dilatation thermique.
\end{definitionbox}

\textbf{Correspondance Notation Tensorielle $\leftrightarrow$ Voigt :}
\begin{center}
\begin{tabular}{ccc|ccc}
\toprule
$C_{1111} \to C_{11}$ & $C_{1122} \to C_{12}$ & $C_{1133} \to C_{13}$ & $C_{1313} \to C_{55}$ & $C_{2323} \to C_{44}$ & $C_{1212} \to C_{66}$ \\
\bottomrule
\end{tabular}
\end{center}

\subsection{Étape 5 : Ansatz de Déplacement}

\begin{theorembox}[Forme des Champs de Déplacement]
\begin{align}
    u_1(x_1, x_2, x_3) &= V_1(x_3) \cos(\delta_1 x_1) \sin(\delta_2 x_2) \\
    u_2(x_1, x_2, x_3) &= V_2(x_3) \sin(\delta_1 x_1) \cos(\delta_2 x_2) \\
    u_3(x_1, x_2, x_3) &= V_3(x_3) \sin(\delta_1 x_1) \sin(\delta_2 x_2)
\end{align}
avec $V_i(x_3) = A_i \cdot e^{\tau x_3}$ où $\tau$ est la \textbf{valeur propre} à déterminer.
\end{theorembox}

La forme de la solution générale pour chaque couche $(i)$ est :
\begin{align}
    U_{\alpha}^{(i)}(x_3) &= \sum_{r=1}^{3} A_{\alpha}^{r(i)} \cdot e^{\tau_r^{(i)} x_3}, \quad \alpha \in \{1, 2, 3\}
\end{align}

\subsection{Étape 6 : Matrice Dynamique $\Gamma(\tau)$ et Valeurs Propres}

L'équation d'équilibre $\text{div}(\boldsymbol{\sigma}) = 0$ conduit au système homogène :
\begin{equation}
    \Gamma(\tau) \cdot \mathbf{A} = \mathbf{0}
\end{equation}

\begin{definitionbox}[Matrice $\Gamma(\tau)$ - Opérateurs $L_{jk}$]
\begin{equation}
\boxed{
    \Gamma(\tau) = \begin{pmatrix}
    L_{11} & L_{12} & L_{13} \\
    L_{21} & L_{22} & L_{23} \\
    L_{31} & L_{32} & L_{33}
    \end{pmatrix}
}
\end{equation}

\textbf{Termes diagonaux :}
\begin{align}
    L_{11} &= C_{55}\tau^2 - (C_{11}\delta_1^2 + C_{66}\delta_2^2) \\
    L_{22} &= C_{44}\tau^2 - (C_{22}\delta_2^2 + C_{66}\delta_1^2) \\
    L_{33} &= C_{33}\tau^2 - (C_{55}\delta_1^2 + C_{44}\delta_2^2)
\end{align}

\textbf{Termes croisés dans le plan (symétriques) :}
\begin{equation}
    L_{12} = L_{21} = -(C_{12} + C_{66})\delta_1\delta_2
\end{equation}

\textbf{Termes hors-plan (ANTISYMÉTRIQUES) :}
\begin{align}
    L_{13} &= +(C_{13} + C_{55})\delta_1\tau, \quad L_{31} = \textcolor{red}{-}(C_{13} + C_{55})\delta_1\tau \\
    L_{23} &= +(C_{23} + C_{44})\delta_2\tau, \quad L_{32} = \textcolor{red}{-}(C_{23} + C_{44})\delta_2\tau
\end{align}
\end{definitionbox}

\begin{warningbox}[Propriété d'Antisymétrie Critique]
$L_{13} = -L_{31}$ et $L_{23} = -L_{32}$ : cette antisymétrie provient de l'équation d'équilibre en direction $x_3$ et est \textbf{essentielle} pour la physique correcte.

L'implémentation de ces opérateurs se trouve dans \texttt{core/mechanical\_pdf.py}, fonctions \texttt{compute\_L\_operators()} et \texttt{get\_Gamma\_matrix()}.
\end{warningbox}

\textbf{Équation Caractéristique :}
\begin{equation}
    \det(\Gamma(\tau)) = 0 \quad \Rightarrow \quad \text{Polynôme d'ordre 6 : } P(\tau) = c_6\tau^6 + c_4\tau^4 + c_2\tau^2 + c_0 = 0
\end{equation}

En posant $X = \tau^2$, on obtient un polynôme cubique avec 3 racines $X_i$. On sélectionne les 3 valeurs propres $\tau_r$ avec $\text{Re}(\tau_r) < 0$ (condition de radiation).

\subsection{Étape 7 : Assemblage $9 \times 9$ et Termes Thermiques}

Le système global pour une couche $(i)$ s'écrit :
\begin{equation}
    \left[ \mathbb{K}_{Dyn}^{(i)} \right]_{(9 \times 9)} \cdot \{ \mathcal{A}^{(i)} \}_{(9 \times 1)} = \{ \mathcal{F}_{Th}^{(i)} \}_{(9 \times 1)}
\end{equation}

\begin{theorembox}[Matrice Bloc-Diagonale $\mathbb{K}_{Dyn}$]
\begin{equation}
\boxed{
    \mathbb{K}_{Dyn}^{(i)} = \begin{pmatrix}
    \Gamma(\tau_1) & 0 & 0 \\
    0 & \Gamma(\tau_2) & 0 \\
    0 & 0 & \Gamma(\tau_3)
    \end{pmatrix}_{9 \times 9}
}
\end{equation}
Chaque bloc $\Gamma(\tau_r)$ est la matrice $3 \times 3$ des opérateurs $L_{jk}$ évaluée à $\tau = \tau_r$.

L'assemblage est implémenté dans \texttt{core/mechanical\_pdf.py}, fonction \texttt{assemble\_K\_dyn\_9x9()}.
\end{theorembox}

\textbf{Vecteur d'amplitudes et vecteur thermique :}
\begin{equation}
    \mathcal{A}^{(i)} = \begin{pmatrix} A_1^{1} \\ A_2^{1} \\ A_3^{1} \\ A_1^{2} \\ A_2^{2} \\ A_3^{2} \\ A_1^{3} \\ A_2^{3} \\ A_3^{3} \end{pmatrix}^{(i)}, \quad
    \mathcal{F}_{Th}^{(i)} = \begin{pmatrix} \mathcal{Q}_1 \\ \mathcal{Q}_2 \\ \mathcal{Q}_3 \\ \mathcal{Q}_1 \\ \mathcal{Q}_2 \\ \mathcal{Q}_3 \\ \mathcal{Q}_1 \\ \mathcal{Q}_2 \\ \mathcal{Q}_3 \end{pmatrix}^{(i)}
\end{equation}

\begin{definitionbox}[Termes de Sollicitation Thermique $\mathcal{Q}_\alpha$]
\begin{align}
    \mathcal{Q}_1^{(i)} &= \left(C_{11}\alpha_{11} + C_{12}\alpha_{22}\right)\delta_1 \, T(x_3 = \bar{h}^{(i)}) \\
    \mathcal{Q}_2^{(i)} &= \left(C_{22}\alpha_{22} + C_{12}\alpha_{11}\right)\delta_2 \, T(x_3 = \bar{h}^{(i)}) \\
    \mathcal{Q}_3^{(i)} &= \left(C_{13}\alpha_{11} + C_{23}\alpha_{22} + C_{33}\alpha_{33}\right) \frac{dT}{dx_3}\bigg|_{x_3 = \bar{h}^{(i)}}
\end{align}

L'implémentation se trouve dans \texttt{core/mechanical\_pdf.py}, fonction \texttt{compute\_Q\_thermal\_vector()}.
\end{definitionbox}

\subsection{Étape 8 : Assemblage Multicouche}

\textbf{Système Global pour $N$ couches :}
\begin{equation}
    \boxed{K_{glob} \cdot \mathbf{C}_{global} = \mathbf{F}_{thermique}}
\end{equation}
avec $6N$ inconnues (18 pour 3 couches).

\textbf{Bilan des Équations :}
\begin{itemize}
    \item 3 équations : Surface libre en $z=0$ ($\sigma_{13} = \sigma_{23} = \sigma_{33} = 0$)
    \item $6(N-1)$ équations : Continuité aux interfaces (déplacements + contraintes)
    \item 3 équations : Surface libre en $z=H$
    \item \textbf{Total :} $3 + 6(N-1) + 3 = 6N$ équations $\checkmark$
\end{itemize}

\begin{warningbox}[Réduction du Nombre d'Équations : 27 $\to$ 18]
\textbf{Formulation théorique complète} (pour $N=3$ couches) :
\begin{center}
\begin{tabular}{lcc}
\toprule
\textbf{Type d'équation} & \textbf{Nombre} & \textbf{Formule} \\
\midrule
Équilibre volumique (div $\sigma$ = 0) & 9 & 3 directions $\times$ 3 couches \\
Continuité aux interfaces & 12 & 2 interfaces $\times$ 6 conditions \\
Conditions aux limites ($\sigma=0$) & 6 & 2 surfaces $\times$ 3 composantes \\
\midrule
\textbf{Total théorique} & \textbf{27} & \\
\bottomrule
\end{tabular}
\end{center}

Les \textbf{9 équations d'équilibre volumique sont IMPLICITEMENT satisfaites} par l'approche modale ! En écrivant la solution avec les valeurs propres $\tau_r$ de $\Gamma(\tau)$, l'équation $\Gamma(\tau_r) \cdot \mathbf{A}^r = 0$ est automatiquement vérifiée.

\textbf{Système final résolu : 18 équations} (conditions d'interface et limites uniquement).

L'assemblage multicouche est implémenté dans \texttt{core/mechanical.py}, fonction \texttt{solve\_multilayer()}.
\end{warningbox}

%==============================================================================
\section{Critères d'Endommagement}
%==============================================================================

\begin{definitionbox}[Indicateur de Dommage D]
\begin{equation}
\boxed{
    D = \max_{ij} \left( \frac{|\sigma_{ij}|}{\sigma_{crit}^{ij}} \right)
}
\end{equation}

\textbf{Interprétation :}
\begin{itemize}
    \item $D < 0.5$ : \textcolor{green}{Zone sûre}
    \item $0.5 \leq D < 0.8$ : \textcolor{orange}{Zone de prudence}
    \item $D \geq 0.8$ : \textcolor{red}{Zone critique - Risque de délamination}
    \item $D \geq 1$ : \textcolor{red}{\textbf{Rupture probable}}
\end{itemize}
\end{definitionbox}

\textbf{Contraintes Critiques par Matériau :}
\begin{center}
\begin{tabular}{lccc}
\toprule
\textbf{Matériau} & $\sigma_t$ (MPa) & $\sigma_c$ (MPa) & $\tau$ (MPa) \\
\midrule
Substrat (Ni) & 1000 & 1200 & 600 \\
BondCoat (MCrAlY) & 500 & 700 & 300 \\
Céramique (YSZ) & 150 & 500 & 120 \\
\bottomrule
\end{tabular}
\end{center}

\textbf{Critère de Tsai-Wu :} Pour matériaux anisotropes :
\begin{equation}
    F = F_3 \sigma_{33} + F_{33} \sigma_{33}^2 + F_{44} \sigma_{23}^2 + F_{55} \sigma_{13}^2 \quad (\text{Rupture si } F \geq 1)
\end{equation}

L'implémentation des critères d'endommagement se trouve dans \texttt{core/damage\_analysis.py}, fonctions \texttt{compute\_damage\_indicator()} et \texttt{compute\_tsai\_wu\_criterion()}.

%==============================================================================
\section{Interprétation Physique des Résultats}
%==============================================================================

\subsection{Signification des Contraintes Transverses}

\begin{center}
\begin{tabular}{lp{10cm}}
\toprule
\textbf{Composante} & \textbf{Interprétation Physique} \\
\midrule
$\sigma_{33}$ (Arrachement) & Contrainte \textbf{normale} à l'interface. \\
& $\sigma_{33} > 0$ : Traction $\to$ Risque de délamination par ouverture (Mode I) \\
& $\sigma_{33} < 0$ : Compression $\to$ Interface en contact, favorable \\
\midrule
$\sigma_{13}, \sigma_{23}$ (Cisaillement) & Contraintes \textbf{tangentielles} aux interfaces. \\
& Responsables du glissement inter-laminaire (Mode II/III) \\
& Pics aux interfaces dus aux discontinuités de $C_{ij}$ et $\alpha$ \\
\bottomrule
\end{tabular}
\end{center}

\subsection{Zones Critiques Typiques}

\begin{itemize}
    \item \textbf{Interface BondCoat/Céramique} : Souvent la plus critique car :
    \begin{itemize}
        \item Fort contraste $C_{ij}$ (180 GPa vs 50 GPa)
        \item Différence de dilatation ($\alpha_{BC} = 14 \times 10^{-6}$ vs $\alpha_{TBC} = 10 \times 10^{-6}$)
    \end{itemize}
    \item \textbf{Bords de la structure} : Effets de bord où les gradients latéraux sont maximaux
\end{itemize}

\subsection{Influence des Paramètres sur les Contraintes}

\begin{center}
\begin{tabular}{lp{8cm}l}
\toprule
\textbf{Action} & \textbf{Effet sur $\sigma_{max}$} & \textbf{Tendance D} \\
\midrule
$\alpha \uparrow$ (TBC plus épais) & Gradient thermique plus étalé, meilleure isolation & $\downarrow$ \\
$L_w \downarrow$ (perturbation courte) & Gradients latéraux plus forts & $\uparrow$ \\
$\Delta T \uparrow$ & Forçage thermique linéairement croissant & $\uparrow$ \\
$\beta \downarrow$ (TBC moins conducteur) & Gradient plus concentré dans TBC & $\uparrow$ dans TBC \\
\bottomrule
\end{tabular}
\end{center}

\subsection{Modes de Rupture aux Interfaces}

\begin{center}
\begin{tikzpicture}[scale=0.7]
    % Mode I - Opening
    \begin{scope}[xshift=0cm]
        \fill[gray!30] (-1.5,0) rectangle (1.5,0.8);
        \fill[blue!20] (-1.5,-0.8) rectangle (1.5,0);
        \draw[thick] (-1.5,0) -- (1.5,0);
        \draw[->, red, very thick] (0,0.5) -- (0,1.3);
        \draw[->, red, very thick] (0,-0.5) -- (0,-1.3);
        \node[below] at (0,-1.8) {\textbf{Mode I}};
        \node[below] at (0,-2.3) {\footnotesize Ouverture};
        \node[below] at (0,-2.7) {\footnotesize $\sigma_{33} > 0$};
    \end{scope}
    
    % Mode II - Shearing
    \begin{scope}[xshift=5cm]
        \fill[gray!30] (-1.5,0) rectangle (1.5,0.8);
        \fill[blue!20] (-1.5,-0.8) rectangle (1.5,0);
        \draw[thick] (-1.5,0) -- (1.5,0);
        \draw[->, red, very thick] (0,0.4) -- (1,0.4);
        \draw[->, red, very thick] (0,-0.4) -- (-1,-0.4);
        \node[below] at (0,-1.8) {\textbf{Mode II}};
        \node[below] at (0,-2.3) {\footnotesize Cisaillement plan};
        \node[below] at (0,-2.7) {\footnotesize $\sigma_{13}$};
    \end{scope}
    
    % Mode III - Tearing
    \begin{scope}[xshift=10cm]
        \fill[gray!30] (-1.5,0) rectangle (1.5,0.8);
        \fill[blue!20] (-1.5,-0.8) rectangle (1.5,0);
        \draw[thick] (-1.5,0) -- (1.5,0);
        \draw[red, very thick] (0,0.4) circle (0.15);
        \fill[red] (0,0.4) circle (0.05);
        \draw[red, very thick] (0,-0.4) circle (0.15);
        \node[red] at (0,-0.4) {$\times$};
        \node[below] at (0,-1.8) {\textbf{Mode III}};
        \node[below] at (0,-2.3) {\footnotesize Cisaillement antiplan};
        \node[below] at (0,-2.7) {\footnotesize $\sigma_{23}$};
    \end{scope}
\end{tikzpicture}
\end{center}

%==============================================================================
\section{Stabilité Numérique}
%==============================================================================

Le système multicouche peut être \textbf{mal conditionné} avec $\text{cond}(K) > 10^{30}$. Plusieurs techniques sont utilisées :

\subsection{Préconditionnement par Équilibrage}

\begin{equation}
    D_r \cdot K_{glob} \cdot D_c \cdot \mathbf{y} = D_r \cdot \mathbf{F}
\end{equation}

où $D_r$ (scaling lignes) et $D_c$ (scaling colonnes) sont des matrices diagonales :
\begin{align}
    D_r[i,i] &= \frac{1}{\max_j |K[i,j]|} \\
    D_c[j,j] &= \frac{1}{\max_i |K_{scaled}[i,j]|}
\end{align}

\subsection{Régularisation de Tikhonov}

Pour $\text{cond}(K) > 10^{10}$, on utilise la décomposition SVD :
\begin{equation}
    K = U \Sigma V^H
\end{equation}

La solution régularisée est :
\begin{equation}
\boxed{
    \mathbf{x}_{reg} = \sum_{i=1}^{n} \frac{\sigma_i^2}{\sigma_i^2 + \lambda^2} \cdot \frac{\mathbf{u}_i^H \mathbf{b}}{\sigma_i} \cdot \mathbf{v}_i
}
\end{equation}

où $\lambda$ est le paramètre de régularisation estimé par GCV (Generalized Cross-Validation).

L'implémentation se trouve dans \texttt{core/mechanical.py}, fonction \texttt{solve\_regularized\_system()}.

%==============================================================================
\section{Comparaison : Méthode Spectrale vs CLT}
%==============================================================================

Le code implémente deux méthodes complémentaires :

\begin{center}
\begin{tabular}{p{6cm}p{6cm}}
\toprule
\textbf{Méthode Spectrale (Principale)} & \textbf{CLT (Classical Laminate Theory)} \\
\midrule
Résout le problème 3D complet & Approximation 2D (hypothèse Kirchhoff) \\
Capture $\sigma_{13}, \sigma_{23}, \sigma_{33}$ & Contraintes planes $\sigma_{11}, \sigma_{22}$ \\
Effets de bord et gradients latéraux & Contraintes uniformes dans le plan \\
Coût calcul : $O(N^3)$ pour $6N \times 6N$ & Coût calcul : $O(N)$ \\
Précis pour structures épaisses & Valide pour $h \ll L$ \\
\bottomrule
\end{tabular}
\end{center}

\textbf{Superposition :} Le code combine les deux méthodes :
\begin{equation}
    \sigma_{total} = \sigma_{CLT} + \sigma_{Spectral}
\end{equation}

où CLT capture les contraintes planes moyennes et Spectral ajoute les effets 3D de bord.

%==============================================================================
\section{Validation : Référence ONERA/Safran}
\label{sec:validation}
%==============================================================================

Les propriétés matériaux utilisées sont issues de publications ONERA/Safran :

\begin{theorembox}[Référence Principale]
\textbf{Bovet, Chiaruttini, Vattré} (ONERA/Safran, 2025)\\
\textit{``Full-scale crystal plasticity modeling and data-driven learning of microstructure effects in polycrystalline turbine blades''}\\
\textbf{Table 3} : Propriétés élastiques de l'Inconel 718
\end{theorembox}

\begin{center}
\begin{tabular}{lcc}
\toprule
\textbf{Propriété} & \textbf{Valeur ONERA} & \textbf{Valeur Code} \\
\midrule
$C_{11}$ (GPa) & 259.6 & 260 \\
$C_{12}$ (GPa) & 179.0 & 179 \\
$C_{44}$ (GPa) & 109.6 & 110 \\
$\alpha$ à RT (K$^{-1}$) & $4.95 \times 10^{-6}$ & $12 \times 10^{-6}$ (moyenne T) \\
\bottomrule
\end{tabular}
\end{center}

\textbf{Plages de validation :}
\begin{itemize}
    \item Contraintes de von Mises typiques FEM : 400--800 MPa
    \item Concentration à la racine de l'aube : jusqu'à 1000 MPa
\end{itemize}

%==============================================================================
\section{Architecture du Code et Interface}
%==============================================================================

\subsection{Structure des Répertoires}

\begin{codebox}[Arborescence du Projet]
\begin{verbatim}
projet-industriel5a/
+-- core/                          # Moteur de calcul
|   +-- mechanical.py              # Solveur spectral principal
|   +-- mechanical_pdf.py          # Solveur selon méthodologie PDF
|   +-- damage_analysis.py         # Critères d'endommagement
|   +-- clt_solver.py              # Théorie classique des stratifiés
|   +-- constants.py               # Propriétés matériaux
|   +-- calculation.py             # Solveur thermique
+-- tabs/                          # Interface Streamlit
|   +-- dashboard_home.py          # Tableau de bord principal
|   +-- mechanical.py              # Onglet analyse mécanique
|   +-- optimization.py            # Onglet optimisation
|   +-- mapping_3d.py              # Visualisation 3D
+-- Profil de temperature Aube.py  # Point d'entrée Streamlit
\end{verbatim}
\end{codebox}

\subsection{Tableau de Correspondance Théorie $\leftrightarrow$ Code}

\begin{center}
\begin{tabular}{|p{3.5cm}|p{4cm}|p{5cm}|}
\hline
\textbf{Étape} & \textbf{Fonction Python} & \textbf{Fichier} \\
\hline
Étapes 1-3: Thermique & \texttt{solve\_tbc\_model\_v2()} & \texttt{core/calculation.py} \\
\hline
Étape 4: Loi Hooke & Constantes \texttt{PROPS\_*} & \texttt{core/constants.py} \\
\hline
Étape 5-6: Matrice $\Gamma$ & \texttt{get\_Gamma\_matrix()} & \texttt{core/mechanical\_pdf.py} \\
\hline
Étape 6: Val. propres & \texttt{solve\_char\_poly()} & \texttt{core/mechanical\_pdf.py} \\
\hline
Étape 7: Termes $Q_\alpha$ & \texttt{compute\_Q\_thermal()} & \texttt{core/mechanical\_pdf.py} \\
\hline
Étape 7: Assemblage 9×9 & \texttt{assemble\_K\_dyn\_9x9()} & \texttt{core/mechanical\_pdf.py} \\
\hline
Étape 8: Multicouche & \texttt{solve\_multilayer()} & \texttt{core/mechanical.py} \\
\hline
Critère D & \texttt{compute\_damage\_indicator()} & \texttt{core/damage\_analysis.py} \\
\hline
Critère Tsai-Wu & \texttt{compute\_tsai\_wu\_criterion()} & \texttt{core/damage\_analysis.py} \\
\hline
\end{tabular}
\end{center}

\subsection{Interface Streamlit}

L'application Streamlit propose plusieurs onglets :

\begin{itemize}
    \item \textbf{Dashboard} (\texttt{tabs/dashboard\_home.py}) : Affiche les KPIs, jauge de risque, radar multi-critères, visualisation 3D du champ thermique, et recommandations automatiques
    \item \textbf{Analyse Mécanique} (\texttt{tabs/mechanical.py}) : Workflow complet d'analyse spectrale avec visualisation des profils de contraintes, indicateurs d'endommagement, et cercles de Mohr
    \item \textbf{Optimisation} (\texttt{tabs/optimization.py}) : Optimisation paramétrique pour minimiser D
    \item \textbf{Visualisation 3D} (\texttt{tabs/mapping\_3d.py}) : Cartographie 3D des contraintes avec Plotly
\end{itemize}

%==============================================================================
\section{Conclusion}
%==============================================================================

Ce rapport démontre la \textbf{traçabilité complète} entre :
\begin{enumerate}
    \item La modélisation mathématique (méthode spectrale, matrice $\Gamma(\tau)$, opérateurs $L_{jk}$)
    \item L'implémentation Python dans \texttt{core/mechanical\_pdf.py}
    \item L'interface utilisateur Streamlit avec visualisations interactives
\end{enumerate}

\textbf{Points clés de l'implémentation :}
\begin{itemize}
    \item \textbf{Méthode numérique robuste} : Identification des coefficients du polynôme caractéristique par évaluation numérique
    \item \textbf{Stabilité numérique} : Préconditionnement par scaling + régularisation Tikhonov pour les systèmes mal conditionnés
    \item \textbf{Validation industrielle} : Propriétés matériaux issues des données ONERA/Safran
    \item \textbf{Critères d'endommagement} : Indicateur D et Tsai-Wu pour identification des zones critiques
    \item \textbf{Superposition CLT+Spectral} : Combinaison des contraintes planes et effets 3D de bord
\end{itemize}

\textbf{Recommandations d'utilisation :}
\begin{itemize}
    \item Maintenir $D < 0.8$ pour les applications critiques
    \item Vérifier que $T_{interface} < T_{crit} = 1100$°C
    \item Augmenter $\alpha$ (épaisseur TBC) pour réduire les contraintes aux interfaces
\end{itemize}

\end{document}
