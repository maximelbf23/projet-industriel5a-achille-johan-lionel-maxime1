\documentclass[11pt,a4paper]{article}
\usepackage[utf8]{inputenc}
\usepackage[T1]{fontenc}
\usepackage[french]{babel}
\usepackage{colortbl}
\usepackage{amsmath,amssymb,amsfonts}
\usepackage{graphicx}
\usepackage{geometry}
\usepackage{tikz}
\usetikzlibrary{arrows.meta, positioning, shapes.geometric, fit, calc, decorations.pathreplacing}
\usepackage{tcolorbox}
\tcbuselibrary{skins,breakable}
\usepackage{fancyhdr}
\usepackage{booktabs}
\usepackage{xcolor}

\geometry{hmargin=2.5cm,vmargin=2.5cm}
\setlength{\headheight}{14pt}

% Couleurs
\definecolor{theoremcolor}{RGB}{59, 130, 246}
\definecolor{definitioncolor}{RGB}{16, 185, 129}
\definecolor{warningcolor}{RGB}{239, 68, 68}
\definecolor{codecolor}{RGB}{100, 116, 139}

% Boîtes
\newtcolorbox{theorembox}[1][]{
    colback=theoremcolor!8,
    colframe=theoremcolor,
    fonttitle=\bfseries,
    title={#1},
    breakable,
    arc=2mm
}
\newtcolorbox{definitionbox}[1][]{
    colback=definitioncolor!8,
    colframe=definitioncolor,
    fonttitle=\bfseries,
    title={#1},
    breakable,
    arc=2mm
}
\newtcolorbox{warningbox}[1][]{
    colback=warningcolor!8,
    colframe=warningcolor,
    fonttitle=\bfseries,
    title={#1},
    breakable,
    arc=2mm
}
\newtcolorbox{keybox}[1][]{
    colback=codecolor!8,
    colframe=codecolor,
    fonttitle=\bfseries,
    title={#1},
    breakable,
    arc=2mm
}

% En-tête
\pagestyle{fancy}
\fancyhf{}
\rhead{Projet IDSA 5A - TBC}
\lhead{Explication : 27 vs 18 Équations}
\rfoot{Page \thepage}

\title{\textbf{Pourquoi 18 Équations et Non 27 ?}\\[0.5cm]
\Large La Réduction du Système par la Méthode Modale\\[0.3cm]
\normalsize Explication Détaillée de l'Optimisation Mathématique}
\author{Documentation Technique - Projet TBC Multicouche}
\date{\today}

\begin{document}

\maketitle

\begin{abstract}
Ce document explique en détail pourquoi l'implémentation numérique du solveur mécanique multicouche utilise \textbf{18 équations} pour un système à 3 couches, alors que le document théorique \textit{Résolution\_équations.pdf} (Étape 8) mentionne \textbf{27 équations}. Cette réduction n'est pas une simplification physique, mais une \textbf{optimisation mathématique} inhérente à la méthode des modes propres.
\end{abstract}

\tableofcontents
\newpage

%==============================================================================
\section{Le Comptage Théorique : 27 Équations}
%==============================================================================

\subsection{Rappel du Problème}

Pour un système multicouche de $N$ couches, nous devons résoudre :
\begin{enumerate}
    \item L'\textbf{équilibre mécanique} dans chaque couche
    \item Les \textbf{conditions de continuité} aux interfaces
    \item Les \textbf{conditions aux limites} sur les surfaces libres
\end{enumerate}

\subsection{Décompte selon le PDF (Étape 8)}

Le document \textit{Résolution\_équations.pdf} compte les équations comme suit pour $N = 3$ couches :

\begin{center}
\begin{tabular}{lcc}
\toprule
\textbf{Type d'équation} & \textbf{Formule} & \textbf{Nombre} \\
\midrule
Équilibre volumique & $N \times 3$ directions & $3 \times 3 = 9$ \\
Continuité aux interfaces & $(N-1) \times 6$ conditions & $2 \times 6 = 12$ \\
Conditions aux limites & $2 \times 3$ (haut + bas) & $2 \times 3 = 6$ \\
\midrule
\textbf{Total} & & \textbf{27} \\
\bottomrule
\end{tabular}
\end{center}

\subsection{Détail des 27 Équations}

\begin{definitionbox}[Les 9 Équations d'Équilibre Volumique]
Pour chaque couche $k \in \{1, 2, 3\}$, l'équilibre statique $\nabla \cdot \boldsymbol{\sigma} = 0$ donne 3 équations :
\begin{align}
\frac{\partial \sigma_{11}}{\partial x_1} + \frac{\partial \sigma_{12}}{\partial x_2} + \frac{\partial \sigma_{13}}{\partial x_3} &= 0 \quad \text{(direction } x_1\text{)} \\
\frac{\partial \sigma_{21}}{\partial x_1} + \frac{\partial \sigma_{22}}{\partial x_2} + \frac{\partial \sigma_{23}}{\partial x_3} &= 0 \quad \text{(direction } x_2\text{)} \\
\frac{\partial \sigma_{31}}{\partial x_1} + \frac{\partial \sigma_{32}}{\partial x_2} + \frac{\partial \sigma_{33}}{\partial x_3} &= 0 \quad \text{(direction } x_3\text{)}
\end{align}
\textbf{Sous-total :} $3 \times 3 = 9$ équations
\end{definitionbox}

\begin{definitionbox}[Les 12 Équations de Continuité aux Interfaces]
À chaque interface entre couches $k$ et $k+1$, on impose :
\begin{itemize}
    \item \textbf{3 continuités de déplacement :} $u_1^{(k)} = u_1^{(k+1)}$, $u_2^{(k)} = u_2^{(k+1)}$, $u_3^{(k)} = u_3^{(k+1)}$
    \item \textbf{3 continuités de traction :} $\sigma_{13}^{(k)} = \sigma_{13}^{(k+1)}$, $\sigma_{23}^{(k)} = \sigma_{23}^{(k+1)}$, $\sigma_{33}^{(k)} = \sigma_{33}^{(k+1)}$
\end{itemize}
\textbf{Sous-total :} $2 \times 6 = 12$ équations (pour 3 couches $\Rightarrow$ 2 interfaces)
\end{definitionbox}

\begin{definitionbox}[Les 6 Conditions aux Limites (Surfaces Libres)]
Sur les surfaces $z = 0$ (bas) et $z = H$ (haut) :
\begin{itemize}
    \item \textbf{En $z = 0$ :} $\sigma_{13} = 0$, $\sigma_{23} = 0$, $\sigma_{33} = 0$
    \item \textbf{En $z = H$ :} $\sigma_{13} = 0$, $\sigma_{23} = 0$, $\sigma_{33} = 0$
\end{itemize}
\textbf{Sous-total :} $2 \times 3 = 6$ équations
\end{definitionbox}

%==============================================================================
\section{La Clé : La Méthode des Modes Propres}
%==============================================================================

\subsection{L'Ansatz Modal}

La méthode spectrale cherche des solutions sous la forme :
\begin{equation}
\boxed{
    u_i(x_1, x_2, x_3) = V_i \cdot e^{\tau x_3} \cdot \sin(\delta_1 x_1) \sin(\delta_2 x_2)
}
\end{equation}

Où :
\begin{itemize}
    \item $V_i$ est l'amplitude du déplacement selon $i$
    \item $\tau$ est la \textbf{valeur propre} à déterminer
    \item $\delta_1, \delta_2$ sont les nombres d'onde latéraux ($= \pi/L_w$)
\end{itemize}

\subsection{Injection dans l'Équation d'Équilibre}

En injectant cet ansatz dans l'équation d'équilibre $\nabla \cdot \boldsymbol{\sigma} = 0$, on obtient un système linéaire homogène :

\begin{theorembox}[Système Matriciel Homogène]
\begin{equation}
\boxed{
    M(\tau) \cdot \mathbf{V} = \mathbf{0}
}
\end{equation}

où $M(\tau)$ est la matrice dynamique $3 \times 3$ :
\begin{equation}
    M(\tau) = \begin{pmatrix}
    C_{55}\tau^2 - K_{11} & -K_{12} & K_{13}\tau \\
    -K_{12} & C_{44}\tau^2 - K_{22} & K_{23}\tau \\
    -K_{13}\tau & -K_{23}\tau & C_{33}\tau^2 - K_{33}
    \end{pmatrix}
\end{equation}
\end{theorembox}

\subsection{La Condition de Non-Trivialité}

Pour que le système ait une solution non-nulle ($\mathbf{V} \neq \mathbf{0}$), il faut :
\begin{equation}
\boxed{
    \det(M(\tau)) = 0
}
\end{equation}

Cette équation caractéristique donne \textbf{6 valeurs propres} $\tau_r$ par couche.

%==============================================================================
\section{Pourquoi les 9 Équations d'Équilibre Disparaissent}
%==============================================================================

\begin{warningbox}[Argument Central]
\textbf{Les 9 équations d'équilibre volumique sont automatiquement satisfaites par la construction modale !}

Voici pourquoi : par définition d'un mode propre, le vecteur $\mathbf{V}_r$ associé à $\tau_r$ appartient au noyau de $M(\tau_r)$ :

\begin{equation}
    M(\tau_r) \cdot \mathbf{V}_r = \mathbf{0} \quad \text{par définition de } \tau_r
\end{equation}

Cela signifie que l'équation d'équilibre est \textbf{identiquement nulle} pour chaque mode propre.
\end{warningbox}

\subsection{Démonstration Pas à Pas}

\begin{enumerate}
    \item \textbf{On cherche $\tau$ tel que} $\det(M(\tau)) = 0$
    
    \item \textbf{Pour chaque racine $\tau_r$}, la matrice $M(\tau_r)$ est singulière (son déterminant est nul)
    
    \item \textbf{Le vecteur propre $\mathbf{V}_r$} est dans le noyau de $M(\tau_r)$, donc :
    \begin{equation}
        M(\tau_r) \cdot \mathbf{V}_r = \mathbf{0}
    \end{equation}
    
    \item \textbf{Or $M(\tau) \cdot \mathbf{V} = \mathbf{0}$} est exactement l'équation d'équilibre sous forme matricielle !
    
    \item \textbf{Conclusion :} L'équilibre est satisfait \textbf{a priori} pour tout mode propre $\tau_r$.
\end{enumerate}

\subsection{Illustration Graphique}

\begin{center}
\begin{tikzpicture}[scale=1.1]
    % Titre
    \node[font=\bfseries] at (0,4) {Approche Théorique Générale};
    \node[font=\bfseries] at (8,4) {Approche Modale (Code)};
    
    % Boîtes gauche
    \draw[thick, fill=red!20, rounded corners] (-2,2) rectangle (2,3.5);
    \node at (0,2.75) {\footnotesize 9 éq. équilibre};
    
    \draw[thick, fill=blue!20, rounded corners] (-2,0) rectangle (2,1.5);
    \node at (0,0.75) {\footnotesize 12 éq. continuité};
    
    \draw[thick, fill=green!20, rounded corners] (-2,-2) rectangle (2,-0.5);
    \node at (0,-1.25) {\footnotesize 6 éq. surfaces libres};
    
    \node at (0,-2.8) {\textbf{Total : 27 éq.}};
    
    % Flèche
    \draw[->, very thick, dashed] (3,1) -- (5,1) node[midway, above] {\footnotesize Réduction};
    \draw[->, very thick, dashed] (3,1) -- (5,1) node[midway, below] {\footnotesize par $\det(M)=0$};
    
    % Boîtes droite - équilibre barré
    \draw[thick, fill=red!10, rounded corners, dashed, opacity=0.5] (6,2) rectangle (10,3.5);
    \node[opacity=0.4] at (8,2.75) {\footnotesize 9 éq. équilibre};
    \draw[red, very thick, opacity=0.7] (6,2) -- (10,3.5);
    \draw[red, very thick, opacity=0.7] (6,3.5) -- (10,2);
    \node[red, font=\scriptsize] at (8,2.2) {Satisfait par $M(\tau_r)\cdot V_r = 0$};
    
    \draw[thick, fill=blue!20, rounded corners] (6,0) rectangle (10,1.5);
    \node at (8,0.75) {\footnotesize 12 éq. continuité};
    
    \draw[thick, fill=green!20, rounded corners] (6,-2) rectangle (10,-0.5);
    \node at (8,-1.25) {\footnotesize 6 éq. surfaces libres};
    
    \node at (8,-2.8) {\textbf{Total : 18 éq.}};
    
    % Annotation
    \draw[decorate, decoration={brace, amplitude=8pt, raise=2pt}] (10.2,1.5) -- (10.2,-2);
    \node[right, font=\footnotesize] at (10.4,-0.25) {$= 6 \times N$ inconnues};
    
\end{tikzpicture}
\end{center}

%==============================================================================
\section{Le Système Final : 18 Équations pour 18 Inconnues}
%==============================================================================

\subsection{Les 18 Inconnues}

Pour chaque couche $k$, nous avons \textbf{6 constantes d'intégration} $C_1^{(k)}, ..., C_6^{(k)}$ correspondant aux 6 modes propres $\tau_1, ..., \tau_6$.

\begin{equation}
\mathbf{C}_{global} = \begin{pmatrix}
C_1^{(1)} \\ \vdots \\ C_6^{(1)} \\[0.5em]
C_1^{(2)} \\ \vdots \\ C_6^{(2)} \\[0.5em]
C_1^{(3)} \\ \vdots \\ C_6^{(3)}
\end{pmatrix}
\in \mathbb{C}^{18}
\end{equation}

\subsection{Les 18 Équations}

\begin{center}
\begin{tabular}{lcc}
\toprule
\textbf{Type} & \textbf{Position} & \textbf{Nombre} \\
\midrule
Surface libre ($z = 0$) & $\sigma_{13} = \sigma_{23} = \sigma_{33} = 0$ & 3 \\
Interface 1-2 & 6 conditions (3 dépl. + 3 contr.) & 6 \\
Interface 2-3 & 6 conditions (3 dépl. + 3 contr.) & 6 \\
Surface libre ($z = H$) & $\sigma_{13} = \sigma_{23} = \sigma_{33} = 0$ & 3 \\
\midrule
\textbf{Total} & & \textbf{18} \\
\bottomrule
\end{tabular}
\end{center}

\subsection{Structure de la Matrice Globale $K_{glob}$}

Le système est écrit sous forme :
\begin{equation}
\boxed{
    K_{glob} \cdot \mathbf{C}_{global} = \mathbf{F}_{thermique}
}
\end{equation}

avec $K_{glob}$ de taille $18 \times 18$ :

\begin{center}
\begin{tikzpicture}[scale=0.55, every node/.style={font=\scriptsize}]
    % Grille principale
    \draw[thick] (0,0) rectangle (9,9);
    
    % Découpage en blocs 6x6
    \draw[dashed] (3,0) -- (3,9);
    \draw[dashed] (6,0) -- (6,9);
    \draw[dashed] (0,3) -- (9,3);
    \draw[dashed] (0,6) -- (9,6);
    
    % Labels colonnes
    \node[below] at (1.5,-0.3) {$\mathbf{C}^{(1)}$};
    \node[below] at (4.5,-0.3) {$\mathbf{C}^{(2)}$};
    \node[below] at (7.5,-0.3) {$\mathbf{C}^{(3)}$};
    
    % Bloc 1: BC z=0 (3 lignes)
    \fill[green!30] (0,8) rectangle (3,9);
    \node at (1.5,8.5) {$B_\sigma \Phi^{(1)}(0)$};
    
    % Bloc 2: Interface 1-2 (6 lignes)
    \fill[blue!20] (0,5) rectangle (3,8);
    \node at (1.5,6.5) {$\Phi^{(1)}(h_1)$};
    \fill[red!20] (3,5) rectangle (6,8);
    \node at (4.5,6.5) {$-\Phi^{(2)}(0)$};
    
    % Bloc 3: Interface 2-3 (6 lignes)
    \fill[blue!20] (3,2) rectangle (6,5);
    \node at (4.5,3.5) {$\Phi^{(2)}(h_2)$};
    \fill[red!20] (6,2) rectangle (9,5);
    \node at (7.5,3.5) {$-\Phi^{(3)}(0)$};
    
    % Bloc 4: BC z=H (3 lignes)
    \fill[orange!30] (6,0) rectangle (9,2);
    \node at (7.5,1) {$B_\sigma \Phi^{(3)}(h_3)$};
    
    % Labels lignes
    \node[left] at (-0.3,8.5) {BC bas (3)};
    \node[left] at (-0.3,6.5) {Interface 1 (6)};
    \node[left] at (-0.3,3.5) {Interface 2 (6)};
    \node[left] at (-0.3,1) {BC haut (3)};
    
    % Dimensions
    \draw[<->] (9.8,0) -- (9.8,9);
    \node[right] at (9.8,4.5) {18 lignes};
    
    \draw[<->] (0,-1.2) -- (9,-1.2);
    \node[below] at (4.5,-1.2) {18 colonnes};
    
    % Légende
    \node at (12,8) {\textcolor{green!60!black}{$\blacksquare$} Surface libre};
    \node at (12,7) {\textcolor{blue!60}{$\blacksquare$} Continuité (+)};
    \node at (12,6) {\textcolor{red!60}{$\blacksquare$} Continuité (-)};
    \node at (12,5) {\textcolor{orange!60}{$\blacksquare$} Surface libre};
\end{tikzpicture}
\end{center}

%==============================================================================
\section{Formule Générale pour N Couches}
%==============================================================================

\subsection{Comptage des Équations}

\begin{theorembox}[Formules Générales]
Pour un système à $N$ couches :

\textbf{Nombre d'inconnues :}
\begin{equation}
    \text{Inconnues} = 6 \times N
\end{equation}

\textbf{Nombre d'équations (approche modale) :}
\begin{equation}
    \text{Équations} = \underbrace{3}_{\text{BC bas}} + \underbrace{6(N-1)}_{\text{interfaces}} + \underbrace{3}_{\text{BC haut}} = 6N
\end{equation}

\textbf{Vérification :} Inconnues $=$ Équations $\checkmark$
\end{theorembox}

\subsection{Comparaison des Approches}

\begin{center}
\begin{tabular}{lccc}
\toprule
\textbf{Approche} & \textbf{Formule} & \textbf{N=3} & \textbf{N=10} \\
\midrule
Théorique (27 éq.) & $9N$ & 27 & 90 \\
Modale (18 éq.) & $6N$ & 18 & 60 \\
\midrule
\textbf{Réduction} & $3N$ équations en moins & 9 & 30 \\
\bottomrule
\end{tabular}
\end{center}

%==============================================================================
\section{L'Implémentation dans le Code Python}
%==============================================================================

\subsection{Fonction d'Assemblage (core/mechanical.py)}

Le code Python exploite cette optimisation dans la fonction \texttt{solve\_multilayer} :

\begin{keybox}[Extrait de core/mechanical.py (lignes 992-1050)]
\begin{verbatim}
def solve_multilayer(layers, delta1, delta2, ...):
    """
    Résout le problème multicouche avec 6N équations.
    
    Les 3N équations d'équilibre sont AUTOMATIQUEMENT
    satisfaites car les modes propres tau_r vérifient:
        det(M(tau_r)) = 0  =>  M(tau_r) @ V_r = 0
    """
    N = len(layers)
    
    # Matrice globale 6N x 6N (pas 9N x 6N !)
    K_glob = np.zeros((6*N, 6*N), dtype=complex)
    
    # Bloc 1: Conditions aux limites en z=0
    K_glob[0:3, 0:6] = B_stress @ Phi_layer0(0)
    
    # Blocs 2 à N: Conditions de continuité
    for k in range(N-1):
        row = 3 + 6*k
        K_glob[row:row+6, 6*k:6*(k+1)] = Phi[k](h[k])
        K_glob[row:row+6, 6*(k+1):6*(k+2)] = -Phi[k+1](0)
    
    # Bloc final: Conditions aux limites en z=H
    K_glob[-3:, -6:] = B_stress @ Phi_layerN(h_N)
    
    # Résolution: 6N équations, 6N inconnues
    C_global = solve(K_glob, F_thermal)
\end{verbatim}
\end{keybox}

\subsection{Vérification de l'Équilibre a Posteriori}

Bien que non nécessaire (l'équilibre est garanti par construction), on peut vérifier :

\begin{keybox}[Test de Validation]
\begin{verbatim}
# Verification que M(tau_r) @ V_r = 0 pour chaque mode
for r, tau_r in enumerate(tau_roots):
    M_r = get_M_matrix(tau_r, delta1, delta2, props)
    V_r = eigenvectors[r]
    residual = M_r @ V_r
    assert np.allclose(residual, 0, atol=1e-10)
    # => L'équilibre est bien satisfait !
\end{verbatim}
\end{keybox}

%==============================================================================
\section{Références dans la Documentation}
%==============================================================================

Cette optimisation est mentionnée dans :

\begin{enumerate}
    \item \textbf{resolution\_matrice.tex} (lignes 144-160) :
    \begin{quote}
    ``Les équations d'équilibre volumique sont automatiquement satisfaites par la propriété des modes propres : $M(\tau_r) \cdot \mathbf{V}_r = \mathbf{0}$ par définition de $\tau_r$.''
    \end{quote}
    
    \item \textbf{implementation\_mecanique.tex} (ligne 281) :
    \begin{quote}
    ``Le système final comporte $6N$ équations et $6N$ inconnues, l'équilibre étant implicitement vérifié.''
    \end{quote}
\end{enumerate}

%==============================================================================
\section{Résumé et Conclusion}
%==============================================================================

\begin{theorembox}[Synthèse]
\begin{center}
\begin{tabular}{lcl}
\textbf{Approche} & \textbf{N=3} & \textbf{Pourquoi ?} \\
\midrule
Théorique générale & 27 éq. & Compte toutes les équations physiques \\
\rowcolor{green!20}
Méthode modale (code) & \textbf{18 éq.} & \textbf{Équilibre implicite via $\det(M)=0$} \\
\end{tabular}
\end{center}

\vspace{0.5cm}

\textbf{Message clé :} La réduction de 27 à 18 équations n'est \underline{pas} une simplification physique, mais une \textbf{optimisation mathématique} qui exploite la propriété fondamentale des modes propres :

\begin{equation}
\boxed{
    \det(M(\tau_r)) = 0 \quad \Leftrightarrow \quad M(\tau_r) \cdot \mathbf{V}_r = \mathbf{0} \quad \Leftrightarrow \quad \text{Équilibre satisfait}
}
\end{equation}
\end{theorembox}

\begin{center}
\begin{tikzpicture}
    \node[draw, rounded corners, fill=definitioncolor!20, text width=12cm, align=center, font=\large, inner sep=10pt] {
        \textbf{Le code est mathématiquement équivalent au PDF.}\\[0.3em]
        Il résout le même problème physique avec moins d'équations\\
        grâce à l'exploitation intelligente de la structure modale.
    };
\end{tikzpicture}
\end{center}

\end{document}
