\documentclass[a4paper,11pt]{article}
\usepackage[utf8]{inputenc}
\usepackage[T1]{fontenc}
\usepackage[french]{babel}
\usepackage{amsmath, amssymb}
\usepackage{geometry}
\usepackage{xcolor}
\usepackage{tcolorbox}
\tcbuselibrary{skins,breakable}
\usepackage{fancyhdr}
\usepackage{enumitem}
\usepackage{booktabs}
\usepackage{hyperref}

\geometry{hmargin=2.5cm,vmargin=2.5cm}
\setlength{\headheight}{14pt}

% Couleurs pour les 4 orateurs
\definecolor{johan}{RGB}{59, 130, 246}
\definecolor{achille}{RGB}{16, 185, 129}
\definecolor{lionel}{RGB}{249, 115, 22}
\definecolor{maxime}{RGB}{139, 92, 246}
\definecolor{codecolor}{RGB}{30, 41, 59}

\newtcolorbox{johanbox}[1][]{colback=johan!5, colframe=johan, fonttitle=\bfseries\large, title={#1}, breakable}
\newtcolorbox{achillebox}[1][]{colback=achille!5, colframe=achille, fonttitle=\bfseries\large, title={#1}, breakable}
\newtcolorbox{lionelbox}[1][]{colback=lionel!5, colframe=lionel, fonttitle=\bfseries\large, title={#1}, breakable}
\newtcolorbox{maximebox}[1][]{colback=maxime!5, colframe=maxime, fonttitle=\bfseries\large, title={#1}, breakable}

\pagestyle{fancy}
\fancyhf{}
\rhead{Projet 5A-IDSA -- Script V3}
\lhead{TBC Multicouches}
\rfoot{Page \thepage}

\title{\textbf{SCRIPT DE SOUTENANCE V3}\\[0.3cm]
\large Distribution Homogene de la Difficulte\\[0.5cm]
Evaluation Thermomecanique des Aubes TBC}
\author{Johan GUITTON $\cdot$ Achille VOISIN $\cdot$ Lionel DAURIAC $\cdot$ Maxime LEBOEUF}
\date{26 janvier 2026}

\begin{document}
\maketitle
\thispagestyle{empty}

\begin{center}
\begin{tcolorbox}[colback=codecolor!10, colframe=codecolor, width=0.95\textwidth]
\centering
\textbf{REPARTITION EQUILIBREE -- 32 SLIDES (10 min chacun)}\\[0.3cm]
\begin{tabular}{|l|c|l|c|}
\hline
\textbf{Orateur} & \textbf{Slides} & \textbf{Contenu} & \textbf{Difficulte} \\
\hline
\textcolor{johan}{Johan} & 1--4, 16--17 & Intro + Matrice $\Gamma(\tau)$ & Facile + Eleve \\
\textcolor{achille}{Achille} & 5--8, 18--19 & TBC/Objectifs + Assemblage 9x9 & Moyen + Eleve \\
\textcolor{lionel}{Lionel} & 9--15 & Methodologie + Spectrale + Hooke & Moyen + Eleve \\
\textcolor{maxime}{Maxime} & 20--32 & Implementation + Resultats & Moyen \\
\hline
\end{tabular}
\end{tcolorbox}
\end{center}

\newpage
\tableofcontents
\newpage

%==============================================================================
\section{JOHAN -- Slides 1-4 puis 16-17 (10 min)}
%==============================================================================

\begin{johanbox}[Slide 1 : Titre (30s)]
<<~Bonjour, je suis Johan Guitton. Avec Achille, Lionel et Maxime, nous presentons notre projet : \textbf{Evaluation Thermomecanique des Zones Critiques d'Endommagement dans les Aubes TBC}.

Methode spectrale implementee en Python/Streamlit, encadrement ONERA (Aurelien Vattre) et ESTACA (Daniel Gaffie).~>>
\end{johanbox}

\begin{johanbox}[Slide 2 : Structure Presentation (30s)]
<<~4 parties : (1) Contexte et Objectifs, (2) Modelisation Mathematique, (3) Implementation Logicielle, (4) Resultats et Validation.

Nous avons reparti les themes pour que chacun presente a la fois du contexte ET du contenu technique.~>>
\end{johanbox}

\begin{johanbox}[Slide 3 : Turbines Haute Pression (1min30)]
<<~Les turbines HP fonctionnent dans un environnement extreme :
\begin{itemize}
    \item Temperatures gaz $>$ 1500\textdegree C
    \item Gradients jusqu'a 200\textdegree C sur quelques mm
    \item Force centrifuge + fatigue cyclique
\end{itemize}
Le defi : augmenter les temperatures pour ameliorer l'efficacite energetique.~>>
\end{johanbox}

\begin{johanbox}[Slide 4 : Architecture TBC (1min30)]
<<~Systeme multicouche TBC :
\begin{itemize}
    \item \textbf{Ceramique YSZ} (50--500 $\mu$m) : isolation thermique principale
    \item \textbf{Bond Coat MCrAlY} ($\sim$10 $\mu$m) : adhesion et protection oxydation
    \item \textbf{Substrat Inconel 718} ($\sim$500 $\mu$m) : resistance mecanique
\end{itemize}
Contraste de proprietes = source des contraintes aux interfaces.~>>
\end{johanbox}

\textit{[Johan enchaine directement sur les slides techniques 16-17]}

\begin{johanbox}[Slide 16 : Matrice Dynamique $\Gamma(\tau)$ (2min) -- TECHNIQUE]
<<~Le coeur de notre methode est la \textbf{matrice dynamique} $\Gamma(\tau)$.

\[
\Gamma(\tau) = \begin{pmatrix}
L_{11} & L_{12} & L_{13} \\
L_{21} & L_{22} & L_{23} \\
-L_{31} & -L_{32} & L_{33}
\end{pmatrix}
\]

\textbf{Point critique} : antisymetrie $L_{13} = -L_{31}$.

Cette propriete provient de l'equilibre en $x_3$ et est essentielle pour la physique correcte.

L'equation caracteristique : $\det(\Gamma(\tau)) = 0$.~>>
\end{johanbox}

\begin{johanbox}[Slide 17 : Resolution Equation Caracteristique (2min)]
<<~$\det(\Gamma(\tau)) = 0$ donne un polynome d'ordre 6.

\textbf{Astuce} : $X = \tau^2$ reduit a un polynome cubique.

$c_6 X^3 + c_4 X^2 + c_2 X + c_0 = 0 \Rightarrow$ 3 racines

\textbf{Selection physique} : $\text{Re}(\tau) < 0$ (condition de radiation).

Seules les ondes evanescentes sont conservees.

Je passe maintenant la parole a Achille.~>>
\end{johanbox}

%==============================================================================
\section{ACHILLE -- Slides 5-8 puis 18-19 (10 min)}
%==============================================================================

\begin{achillebox}[Slide 5 : Mecanismes de Defaillance (1min30)]
<<~Merci Johan. Les systemes TBC presentent des modes de defaillance critiques :

\textbf{Delamination} : perte d'adhesion interface ceramique/bond coat (contrainte $\sigma_{33}$).

\textbf{Ecaillage} : perte de morceaux de revetement.

\textbf{Fissuration} : propagation par concentration de contraintes.

Impact : cause principale de defaillance et couts de maintenance eleves.~>>
\end{achillebox}

\begin{achillebox}[Slide 6 : Objectifs du Projet (1min)]
<<~4 objectifs :
\begin{enumerate}
    \item \textbf{Modeliser} la reponse thermomecanique 3D
    \item \textbf{Predire} les zones critiques aux interfaces
    \item \textbf{Quantifier} les effets des parametres cles
    \item \textbf{Fournir} des cartes de sensibilite pour la conception
\end{enumerate}~>>
\end{achillebox}

\begin{achillebox}[Slide 7 : Partenaires (45s)]
<<~Partenaires :
\begin{itemize}
    \item \textbf{ONERA} : expertise mecanique, supervision Aurelien Vattre
    \item \textbf{Safran} : contexte industriel, donnees materiaux
    \item \textbf{ESTACA} : encadrement pedagogique, Daniel Gaffie
\end{itemize}~>>
\end{achillebox}

\begin{achillebox}[Slide 8 : Planning du Projet (45s)]
<<~Projet sur 6-8 mois :
\begin{itemize}
    \item Phase 1 (Mois 1-2) : Bibliographie, etude methode spectrale
    \item Phase 2 (Mois 2-4) : Implementation Python solveurs
    \item Phase 3 (Mois 4-6) : Interface Streamlit, Plotly 3D
    \item Phase 4 (Mois 6-8) : Tests croises, rapport
\end{itemize}~>>
\end{achillebox}

\textit{[Achille enchaine sur les slides techniques 18-19]}

\begin{achillebox}[Slide 18 : Assemblage 9x9 (2min) -- TECHNIQUE]
<<~On assemble un systeme lineaire global $9 \times 9$ :

\[
[K_{Dyn}] \cdot \{A\} = \{F_{Th}\}
\]

$K_{Dyn}$ est \textbf{bloc-diagonale} : 3 blocs $\Gamma(\tau_r)$.

$F_{Th}$ contient les termes sources thermiques $Q_\alpha$.

Resolution par inversion matricielle pour les 9 amplitudes.~>>
\end{achillebox}

\begin{achillebox}[Slide 19 : Assemblage Multicouche (2min)]
<<~Pour N couches, systeme global avec :
\begin{itemize}
    \item Conditions limites : surface libre ($\sigma \cdot n = 0$)
    \item Conditions continuite : $u$ et $\sigma$ aux interfaces
\end{itemize}

\textbf{Reduction} : 27 equations theoriques $\rightarrow$ \textbf{18 equations} spectrales.

Gain de 9 equations car l'equilibre volumique est implicitement satisfait.

Je laisse la parole a Lionel.~>>
\end{achillebox}

%==============================================================================
\section{LIONEL -- Slides 9-15 (10 min)}
%==============================================================================

\begin{lionelbox}[Slide 9 : Methodologie Generale (1min)]
<<~Merci Achille. Voici le flux de calcul semi-analytique :

\textbf{Entrees} $\rightarrow$ \textbf{Thermique} (Fourier) $\rightarrow$ \textbf{Mecanique} ($\Gamma(\tau)$) $\rightarrow$ \textbf{Dommage} (D, Tsai-Wu) $\rightarrow$ \textbf{Sorties}

Nos parametres d'entree : $\alpha$, $L_w$, $\beta$, proprietes materiaux.~>>
\end{lionelbox}

\begin{lionelbox}[Slide 10 : Livrables (1min)]
<<~3 livrables :
\begin{enumerate}
    \item \textbf{Module Python} : $\sim$3000 lignes, 5 modules dans \texttt{core/}
    \item \textbf{Interface Streamlit} : 8 onglets, dashboard KPIs, 3D Plotly
    \item \textbf{Documentation} : rapport synthese, tracabilite theorie-code
\end{enumerate}~>>
\end{lionelbox}

\begin{lionelbox}[Slide 11 : Methode Spectrale (1min30) -- TECHNIQUE]
<<~La methode spectrale decompose les champs en series de Fourier :

\[
T(x) = \sum T_{mn} \sin(\delta x)
\]

\textbf{Avantages vs FEM} :
\begin{itemize}
    \item Semi-analytique : pas de discretisation
    \item Rapidite : resolution directe
    \item Precision : erreurs minimales
\end{itemize}~>>
\end{lionelbox}

\begin{lionelbox}[Slide 12 : Representation Spectrale Temperature (1min)]
<<~Le champ de temperature :
\[
T(x_1, x_2, x_3) = \sum T_{mn}(x_3) \sin(\delta_1 x_1) \sin(\delta_2 x_2)
\]

$\delta = \pi/L_w$ est le nombre d'onde.

$T_{mn}(x_3)$ ne depend que de la profondeur : simplification 3D $\rightarrow$ 1D.~>>
\end{lionelbox}

\begin{lionelbox}[Slide 13 : Solution Thermique par Couche (1min)]
<<~Dans chaque couche $i$, solution exponentielle :

\[
T^{(i)}(x_3) = A^{(i)} e^{\lambda^{(i)} x_3} + B^{(i)} e^{-\lambda^{(i)} x_3}
\]

$\lambda^{(i)} = \delta_\eta \sqrt{k_{\eta\eta}^{(i)}/k_{33}^{(i)}}$

2N inconnues (A, B) determinees par continuite aux interfaces.~>>
\end{lionelbox}

\begin{lionelbox}[Slide 14 : Loi de Hooke Thermo-Elastique (1min30) -- TECHNIQUE]
<<~Formulation mecanique :
\[
\sigma_{ij} = C_{ijkl} (\varepsilon_{kl} - \alpha_{kl} \Delta T)
\]

Proprietes par couche (donnees ONERA/Safran) :
\begin{center}
\begin{tabular}{lcc}
Materiau & $C_{11}$ & $\alpha$ \\
\hline
Substrat & 260 GPa & 13 $\mu\varepsilon$/K \\
Bond Coat & 180 GPa & 14 $\mu\varepsilon$/K \\
Ceramique & 50 GPa & 10 $\mu\varepsilon$/K \\
\end{tabular}
\end{center}

Le contraste important (260 vs 50 GPa) genere les contraintes.~>>
\end{lionelbox}

\begin{lionelbox}[Slide 15 : Ansatz de Deplacement Modal (1min)]
<<~Forme supposee des deplacements :

$u_i = V_i(x_3) \times \text{trig}(\delta x_1, \delta x_2)$

$V_i(x_3) = A_i \exp(\tau x_3)$

\textbf{Condition de radiation} : $\text{Re}(\tau) < 0$ assure la stabilite.

Je passe la parole a Maxime pour l'implementation et les resultats.~>>
\end{lionelbox}

%==============================================================================
\section{MAXIME -- Slides 20-32 (10 min)}
%==============================================================================

\begin{maximebox}[Slide 20 : Criteres d'Endommagement (1min)]
<<~Merci Lionel. Les criteres de dommage :

\textbf{Indicateur D} : $D = \max(|\sigma_{ij}|/\sigma_{crit}^{ij})$

$D < 0.5$ : Sur. $0.5 \leq D < 0.8$ : Prudence. $D \geq 0.8$ : Critique.

\textbf{Tsai-Wu} pour materiaux anisotropes.~>>
\end{maximebox}

\begin{maximebox}[Slide 21 : Stabilite Numerique (1min)]
<<~Defi : conditionnement peut depasser $10^{30}$.

\textbf{Solutions} :
\begin{itemize}
    \item Preconditionnement $K' = D_r \cdot K \cdot D_c$
    \item Regularisation Tikhonov (SVD)
    \item Normalisation $C_{ref} = 200$ GPa
\end{itemize}

Precision atteinte : $10^{-8}$.~>>
\end{maximebox}

\begin{maximebox}[Slide 22 : Architecture Logicielle (1min)]
<<~Repertoire \texttt{core/} :
\begin{itemize}
    \item \texttt{mechanical\_pdf.py} : Solveur Spectral (1031 lignes)
    \item \texttt{mechanical.py} : Assemblage Multicouche (1507 lignes)
    \item \texttt{damage\_analysis.py} : Criteres (367 lignes)
    \item \texttt{calculation.py} : Solveur Thermique (269 lignes)
    \item \texttt{constants.py} : Donnees Materiaux
\end{itemize}

Total : $\sim$3000 lignes Python.~>>
\end{maximebox}

\begin{maximebox}[Slide 23 : Dashboard Streamlit (1min)]
<<~Interface avec :
\begin{itemize}
    \item 6 KPI Cards temps reel
    \item Jauge de risque, radar multi-criteres
    \item Visualisation 3D Plotly interactive
    \item Badge "Conforme ONERA"
\end{itemize}~>>
\end{maximebox}

\begin{maximebox}[Slide 24 : Resultats Thermiques (45s)]
<<~Profil thermique T(z) :
\begin{itemize}
    \item Gradient substrat : $\sim$20\textdegree C/mm
    \item Gradient ceramique : $\sim$400\textdegree C/mm
    \item $T_{interface} \approx 1050$\textdegree C (marge 50\textdegree C vs $T_{crit}$)
\end{itemize}~>>
\end{maximebox}

\begin{maximebox}[Slide 25 : Resultats Mecaniques (1min)]
<<~Profils de contraintes :
\begin{itemize}
    \item $\sigma_{33}$ : pic $\sim$200 MPa a l'interface BC/Ceramique
    \item $\sigma_{13}$ : pic $\sim$250 MPa
    \item Risque de delamination si $\sigma_{33} > \sigma_{crit}$
\end{itemize}~>>
\end{maximebox}

\begin{maximebox}[Slide 26 : Etudes Parametriques (45s)]
<<~Sensibilite :
\begin{itemize}
    \item $\alpha \uparrow \Rightarrow D \downarrow$ (benefique)
    \item $L_w \downarrow \Rightarrow D \uparrow$ (critique)
    \item Recommandation : $\alpha \geq 0.2$
\end{itemize}~>>
\end{maximebox}

\begin{maximebox}[Slide 27 : Validation ONERA (45s)]
<<~Ecarts $< 1\%$ sur modules elastiques vs reference ONERA.

Contraintes dans plage FEM (400--800 MPa).

$\Rightarrow$ Modele valide.~>>
\end{maximebox}

\begin{maximebox}[Slide 28 : Demo Application (30s)]
<<~Workflow : Parametrage $\rightarrow$ Calcul ($<$1s) $\rightarrow$ Visualisation $\rightarrow$ Export PDF/CSV.~>>
\end{maximebox}

\begin{maximebox}[Slide 29 : Limites (30s)]
<<~Limites :
\begin{itemize}
    \item Geometrie plane (vs courbure reelle)
    \item Thermique stationnaire
    \item Elastique lineaire
\end{itemize}

Validite : $T < 1100$\textdegree C, $\sigma < 500$ MPa.~>>
\end{maximebox}

\begin{maximebox}[Slide 30 : Perspectives (30s)]
<<~Evolutions :
\begin{itemize}
    \item Geometrie cylindrique
    \item Fatigue thermique (cyclage)
    \item Machine Learning (surrogate)
    \item Materiaux avances (CMC)
\end{itemize}~>>
\end{maximebox}

\begin{maximebox}[Slide 31 : Conclusion (30s)]
<<~\textbf{Objectifs atteints} : modelisation + prediction zones critiques.

\textbf{Outil valide} : Python/Streamlit + validation ONERA.

\textbf{Tracabilite} : Theorie $\leftrightarrow$ Code.

\textbf{Potentiel industriel} : aide conception rapide.~>>
\end{maximebox}

\begin{maximebox}[Slide 32 : Remerciements (30s)]
<<~Merci a :
\begin{itemize}
    \item ONERA : Aurelien Vattre
    \item ESTACA : Daniel Gaffie
\end{itemize}

Equipe : Johan, Achille, Lionel, Maxime.

\begin{center}
\textbf{MERCI DE VOTRE ATTENTION -- Questions ?}
\end{center}~>>
\end{maximebox}

\end{document}
