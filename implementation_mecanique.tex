\documentclass[11pt,a4paper]{article}
\usepackage[utf8]{inputenc}
\usepackage[T1]{fontenc}
\usepackage[french]{babel}
\usepackage{amsmath,amssymb,amsfonts}
\usepackage{geometry}
\usepackage{graphicx}
\usepackage{xcolor}
\usepackage{listings}
\usepackage{hyperref}
\usepackage{booktabs}
\usepackage{fancyhdr}
\usepackage{tcolorbox}

\geometry{margin=2.5cm}

% Couleurs
\definecolor{codeblue}{RGB}{59,130,246}
\definecolor{codegray}{RGB}{100,116,139}
\definecolor{codegreen}{RGB}{16,185,129}
\definecolor{codebg}{RGB}{248,250,252}

% Style listings Python
\lstdefinestyle{python}{
    backgroundcolor=\color{codebg},
    basicstyle=\ttfamily\small,
    keywordstyle=\color{codeblue}\bfseries,
    commentstyle=\color{codegray}\itshape,
    stringstyle=\color{codegreen},
    breaklines=true,
    frame=single,
    rulecolor=\color{codegray!30},
    numbers=left,
    numberstyle=\tiny\color{codegray},
    language=Python
}

% En-tête
\pagestyle{fancy}
\fancyhf{}
\fancyhead[L]{\textit{Projet Multicouche}}
\fancyhead[R]{\textit{Documentation Technique}}
\fancyfoot[C]{\thepage}

\title{
    \vspace{-1cm}
    \textbf{Implémentation du Solveur Mécanique Spectral}\\[0.5em]
    \large Correspondance entre les Équations Théoriques et le Code Python
}
\author{Documentation Technique - Projet TBC Multicouche}
\date{Janvier 2026}

\begin{document}

\maketitle

\begin{abstract}
Ce document présente l'implémentation informatique du modèle mécanique spectral pour l'analyse thermoélastique de plaques multicouches. Il établit la correspondance rigoureuse entre les équations théoriques du document \textit{Résolution\_équations.pdf} et leur traduction dans le code Python \texttt{core/mechanical.py}.
\end{abstract}

\tableofcontents
\newpage

%==============================================================================
\section{Introduction}
%==============================================================================

Le solveur mécanique implémenté résout le problème d'équilibre thermoélastique d'une plaque multicouche soumise à un chargement thermique. La méthode utilisée est basée sur une décomposition spectrale (séries de Fourier) couplée à une résolution modale dans l'épaisseur.

\subsection{Système de Coordonnées}

\begin{itemize}
    \item $x_1, x_2$ : directions dans le plan de la plaque
    \item $x_3$ : direction normale (à travers l'épaisseur)
    \item $\delta_1 = \pi/L_w$, $\delta_2 = \pi/L_w$ : nombres d'onde spectraux
\end{itemize}

\subsection{Ansatz de Déplacement}

Les champs de déplacement sont développés sous forme modale :
\begin{equation}
    U_j^{(k)}(x_3) = \sum_{r=1}^{6} A_r^{(k)} V_j^{(k,r)} e^{\tau_r^{(k)} x_3}
    \label{eq:ansatz}
\end{equation}

où $\tau_r$ sont les valeurs propres (modes) et $V_j^{(r)}$ les vecteurs propres de déplacement.

%==============================================================================
\section{Équations d'Équilibre Volumique}
%==============================================================================

Pour chaque couche $k$, l'équilibre statique impose $\mathrm{div}(\boldsymbol{\sigma}) = 0$, projeté selon les trois directions.

\subsection{Équation 1 : Projection sur $x_1$}

\begin{tcolorbox}[title=Forme Théorique (PDF Eq. 1), colback=blue!5, colframe=codeblue]
\begin{equation}
    \left(C_{55}\tau_r^2 - [C_{11}\delta_1^2 + C_{66}\delta_2^2]\right) V_1^{(r)} 
    - (C_{12} + C_{66})\delta_1\delta_2 \, V_2^{(r)} 
    + (C_{13} + C_{55})\delta_1\tau_r \, V_3^{(r)} = F_{\text{th},1}
\end{equation}
\end{tcolorbox}

\textbf{Implémentation Python} (lignes 35-38 de \texttt{get\_M\_matrix}) :

\begin{lstlisting}[style=python]
# Ligne 1 de la matrice M(tau)
M[0, 0] = C55 * tau**2 - K11          # K11 = C11*delta1^2 + C66*delta2^2
M[0, 1] = -K12                         # K12 = (C12+C66)*delta1*delta2
M[0, 2] = K13_coeff * tau              # K13_coeff = (C13+C55)*delta1
\end{lstlisting}

\begin{table}[h]
\centering
\begin{tabular}{lcc}
\toprule
\textbf{Terme} & \textbf{Formule Théorique} & \textbf{Code Python} \\
\midrule
$M_{11}$ & $C_{55}\tau^2 - (C_{11}\delta_1^2 + C_{66}\delta_2^2)$ & \texttt{C55*tau**2 - K11} \\
$M_{12}$ & $-(C_{12} + C_{66})\delta_1\delta_2$ & \texttt{-K12} \\
$M_{13}$ & $+(C_{13} + C_{55})\delta_1\tau$ & \texttt{+K13\_coeff*tau} \\
\bottomrule
\end{tabular}
\caption{Correspondance Théorie/Code pour l'Équation 1}
\end{table}

%------------------------------------------------------------------------------
\subsection{Équation 2 : Projection sur $x_2$}

\begin{tcolorbox}[title=Forme Théorique (PDF Eq. 2), colback=blue!5, colframe=codeblue]
\begin{equation}
    -(C_{12} + C_{66})\delta_1\delta_2 \, V_1^{(r)} 
    + \left(C_{44}\tau_r^2 - [C_{66}\delta_1^2 + C_{22}\delta_2^2]\right) V_2^{(r)} 
    + (C_{23} + C_{44})\delta_2\tau_r \, V_3^{(r)} = F_{\text{th},2}
\end{equation}
\end{tcolorbox}

\textbf{Implémentation Python} (lignes 40-43) :

\begin{lstlisting}[style=python]
# Ligne 2 de la matrice M(tau)
M[1, 0] = -K12                         # Symetrie: M21 = M12
M[1, 1] = C44 * tau**2 - K22           # K22 = C66*delta1^2 + C22*delta2^2
M[1, 2] = K23_coeff * tau              # K23_coeff = (C23+C44)*delta2
\end{lstlisting}

%------------------------------------------------------------------------------
\subsection{Équation 3 : Projection sur $x_3$ (Arrachement)}

\begin{tcolorbox}[title=Forme Théorique (PDF Eq. 3), colback=red!5, colframe=red!70!black]
\begin{equation}
    \boxed{-(C_{13} + C_{55})\delta_1\tau_r} \, V_1^{(r)} 
    \boxed{-(C_{23} + C_{44})\delta_2\tau_r} \, V_2^{(r)} 
    + \left(C_{33}\tau_r^2 - [C_{55}\delta_1^2 + C_{44}\delta_2^2]\right) V_3^{(r)} = F_{\text{th},3}
\end{equation}
\end{tcolorbox}

\textbf{Note importante} : Les termes encadrés ont des \textbf{signes négatifs}. Cette correction a été appliquée dans le code.

\textbf{Implémentation Python} (lignes 45-48) :

\begin{lstlisting}[style=python]
# Ligne 3 de la matrice M(tau)
# CORRECTION: Signes NEGATIFS selon Eq. 3 du PDF
M[2, 0] = -K13_coeff * tau   # -(C13+C55)*delta1*tau
M[2, 1] = -K23_coeff * tau   # -(C23+C44)*delta2*tau
M[2, 2] = C33 * tau**2 - K33 # K33 = C55*delta1^2 + C44*delta2^2
\end{lstlisting}

%==============================================================================
\section{Matrice Dynamique $M(\tau)$ Complète}
%==============================================================================

La matrice $M(\tau)$ 3×3 s'écrit sous forme matricielle :

\begin{equation}
M(\tau) = \begin{pmatrix}
C_{55}\tau^2 - K_{11} & -K_{12} & +K_{13}\tau \\[0.5em]
-K_{12} & C_{44}\tau^2 - K_{22} & +K_{23}\tau \\[0.5em]
\textcolor{red}{-K_{13}\tau} & \textcolor{red}{-K_{23}\tau} & C_{33}\tau^2 - K_{33}
\end{pmatrix}
\label{eq:M_matrix}
\end{equation}

avec les coefficients :
\begin{align}
    K_{11} &= C_{11}\delta_1^2 + C_{66}\delta_2^2 \\
    K_{12} &= (C_{12} + C_{66})\delta_1\delta_2 \\
    K_{13} &= (C_{13} + C_{55})\delta_1 \\
    K_{22} &= C_{66}\delta_1^2 + C_{22}\delta_2^2 \\
    K_{23} &= (C_{23} + C_{44})\delta_2 \\
    K_{33} &= C_{55}\delta_1^2 + C_{44}\delta_2^2
\end{align}

%==============================================================================
\section{Résolution de l'Équation Caractéristique}
%==============================================================================

Les valeurs propres $\tau_r$ sont obtenues en résolvant :
\begin{equation}
    \det(M(\tau)) = 0
\end{equation}

Le déterminant est un polynôme de degré 6 en $\tau$ (pair, donc cubique en $\tau^2$) :
\begin{equation}
    P(\tau^2) = c_6 \tau^6 + c_4 \tau^4 + c_2 \tau^2 + c_0 = 0
\end{equation}

\textbf{Implémentation} (\texttt{solve\_characteristic\_equation}, lignes 78-186) :

\begin{lstlisting}[style=python]
# Coefficient c6 (analytique exact)
c6 = props['C55'] * props['C44'] * props['C33']

# Coefficients c4, c2, c0 par evaluation numerique
P_0 = get_det_at_X(0)   # P(0) = c0
P_1 = get_det_at_X(1)   # P(1) = c6 + c4 + c2 + c0
P_2 = get_det_at_X(2)   # P(2) = 8c6 + 4c4 + 2c2 + c0

# Resolution du systeme lineaire pour c4, c2
b1 = P_1 - c6 - c0
b2 = P_2 - 8*c6 - c0
c4 = (b2 - 2*b1) / 2
c2 = b1 - c4

# Racines en X = tau^2
X_roots = np.roots([1, c4/c6, c2/c6, c0/c6])

# 6 racines tau = +/- sqrt(X)
tau_roots = [+sqrt(X), -sqrt(X) for X in X_roots]
\end{lstlisting}

%==============================================================================
\section{Vecteurs Propres de Déplacement et Contrainte}
%==============================================================================

\subsection{Vecteur Propre de Déplacement $V_r$}

Pour chaque valeur propre $\tau_r$, le vecteur propre $V_r \in \ker(M(\tau_r))$ est calculé par la méthode des cofacteurs.

\textbf{Normalisation} : $V_3 = 1$ (convention du PDF)

\subsection{Matrice $R(\tau)$ et Vecteur Contrainte $W_r$}

La matrice $R(\tau)$ relie les déplacements aux contraintes transverses :
\begin{equation}
    W_r = R(\tau_r) \cdot V_r = \begin{pmatrix} \sigma_{13} \\ \sigma_{23} \\ \sigma_{33} \end{pmatrix}
\end{equation}

avec :
\begin{equation}
R(\tau) = \begin{pmatrix}
C_{55}\tau & 0 & C_{55}\delta_1 \\
0 & C_{44}\tau & C_{44}\delta_2 \\
-C_{13}\delta_1 & -C_{23}\delta_2 & C_{33}\tau
\end{pmatrix}
\end{equation}

%==============================================================================
\section{Système Global et Conditions aux Limites}
%==============================================================================

Le système de 27 équations (pour 3 couches) se décompose en :

\begin{table}[h]
\centering
\begin{tabular}{clcc}
\toprule
\textbf{Bloc} & \textbf{Description} & \textbf{Nb Eq.} & \textbf{Lignes} \\
\midrule
1-3 & Équilibre volumique (3 couches × 3 dir.) & 9 & Satisfait par $\det(M)=0$ \\
4 & Interface 1-2 : Continuité $U$ et $\sigma$ & 6 & 1-6 de $K_{\text{glob}}$ \\
5 & Interface 2-3 : Continuité $U$ et $\sigma$ & 6 & 7-12 de $K_{\text{glob}}$ \\
6 & Bord bas ($z=0$) : Surface libre & 3 & 13-15 de $K_{\text{glob}}$ \\
7 & Bord haut ($z=H$) : Surface libre & 3 & 16-18 de $K_{\text{glob}}$ \\
\bottomrule
\end{tabular}
\caption{Structure du système d'équations}
\end{table}

\textbf{Note} : L'implémentation utilise une formulation 6N (18 équations pour 3 couches) car les 9 équations d'équilibre sont automatiquement satisfaites par la construction modale.

%==============================================================================
\section{Conclusion}
%==============================================================================

L'implémentation dans \texttt{core/mechanical.py} respecte rigoureusement les équations du document \textit{Résolution\_équations.pdf}. La correction des signes de $M_{31}$ et $M_{32}$ (termes de couplage $U_3$-$U_1$ et $U_3$-$U_2$) assure la conformité avec l'équation 3 du PDF.

La méthode modale utilisée est mathématiquement équivalente au système explicite 27×27 mais plus efficace numériquement car elle exploite la structure propre du problème.

\end{document}
