\documentclass[a4paper,11pt]{article}
\usepackage[utf8]{inputenc}
\usepackage[T1]{fontenc}
\usepackage[french]{babel}
\usepackage{amsmath}
\usepackage{amssymb}
\usepackage{geometry}
\usepackage{hyperref}

\geometry{hmargin=2.5cm,vmargin=2.5cm}

\title{Synthèse de l'Implémentation du Modèle Mécanique (TBC)}
\author{Projet Industriel 5A - Modélisation Thermomécanique}
\date{14 Décembre 2025}

\begin{document}

\maketitle

\section*{Contexte}
Ce document résume les travaux techniques réalisés pour intégrer la résolution mécanique (calcul des modes propres) dans l'application de simulation, conformément aux spécifications des documents \texttt{resolution\_mécanique\_5A.pdf} et \texttt{ProjectEstaca.pdf}.

\hrule
\vspace{0.5cm}

\section{Objectif}

L'objectif était d'implémenter l'étape 7 de la résolution semi-analytique : \textbf{Trouver les racines de l'équation caractéristique mécanique}.
Ces racines ($\tau$) correspondent aux modes propres qui régissent la décroissance des contraintes dans l'épaisseur de la gaine.

\section{Implémentation réalisée}

\subsection{Définition des Propriétés Matériaux (\texttt{core/constants.py})}
Nous avons ajouté les propriétés d'un matériau orthotrope générique (type céramique poreuse/Zircone) sous la forme d'une matrice de rigidité $C_{ij}$.
\begin{itemize}
    \item \textbf{Fichier :} \texttt{core/constants.py}
    \item \textbf{Données :} Dictionnaire \texttt{MECHANICAL\_PROPS} contenant les 9 constantes élastiques ($C_{11}$ à $C_{66}$).
\end{itemize}

\subsection{Cœur de Calcul (\texttt{core/mechanical.py})}
C'est le module central qui effectue la résolution numérique.

\paragraph{Algorithme implémenté :}
\begin{enumerate}
    \item \textbf{Construction de la Matrice Dynamique $M(\tau)$} :\\
    Implémentation de la matrice $3 \times 3$ dérivée des équations d'équilibre (section 6 du PDF). Elle dépend de la variable spectrale $\tau$ et des nombres d'onde $\delta_1, \delta_2$.
    
    \item \textbf{Calcul du Déterminant (Pivot de Gauss)} :\\
    Comme demandé, nous n'utilisons pas une boîte noire mais une méthode explicite. La fonction \texttt{compute\_determinant\_gaussian} effectue une élimination de Gauss pour calculer la valeur du déterminant pour un $\tau$ donné.

    \item \textbf{Résolution de l'Équation Caractéristique} :\\
    L'équation est de la forme $\det(M(\tau)) = 0$. Théoriquement, c'est un polynôme bicubique en $\tau$ :
    $$ P(\tau^2) = c_6 (\tau^2)^3 + c_4 (\tau^2)^2 + c_2 (\tau^2) + c_0 = 0 $$
    
    Plutôt que de développer le déterminant symboliquement (très lourd), nous utilisons une \textbf{approche numérique robuste} :
    \begin{itemize}
        \item Nous évaluons le déterminant pour 3 valeurs tests de $X = \tau^2$ (0, 1, 2).
        \item Nous résolvons un système linéaire simple pour identifier les coefficients $c_4, c_2, c_0$ (sachant que $c_6 = C_{33}C_{44}C_{55}$).
    \end{itemize}
    
    \item \textbf{Normalisation et Racines} :\\
    Pour garantir la précision numérique (les coefficients atteignant $10^{33}$), nous normalisons le polynôme par $c_6$ avant de chercher les racines avec \texttt{numpy.roots}.
\end{enumerate}

\subsection{Intégration Interface (\texttt{Profil de température Aube.py})}
Un nouvel onglet \textbf{"⚙️ Calcul Mécanique"} a été ajouté à l'application.
\begin{itemize}
    \item Il récupère dynamiquement le paramètre $L_w$ (longueur d'onde) défini par l'utilisateur.
    \item Il lance le calcul en temps réel.
    \item Il affiche le polynôme caractéristique et les 6 racines $\tau$ sous forme tabulaire.
\end{itemize}

\subsection{Calcul des Vecteurs Propres (Etape 8)}
Cette étape permet de déterminer la forme modale des déplacements dans l'épaisseur de la plaque pour chaque mode $\tau$.

\paragraph{Justification de l'implémentation :}
Le système $M(\tau)\mathbf{V} = 0$ impliquant une matrice singulière (car $\det(M(\tau)) \approx 0$), une inversion directe est impossible.
L'algorithme implémenté (fonction \texttt{compute\_eigenvector}) utilise la \textbf{méthode des cofacteurs} :
\begin{itemize}
    \item Les composantes du vecteur propre $\mathbf{V}$ sont calculées comme les cofacteurs de la dernière ligne de $M(\tau)$ (ce qui revient à effectuer le produit vectoriel des deux premières lignes colonnes).
    \item Cette approche garantit une solution non triviale robuste numériquement.
    \item Le vecteur est ensuite normalisé par la condition $V_3 = 1$ (ou par sa norme si $V_3 \approx 0$) pour assurer l'unicité de la solution.
\end{itemize}

\section{Résultats et Vérification}

Les tests effectués (\texttt{test\_mechanical.py} et via l'interface) confirment la validité physique des résultats :
\begin{itemize}
    \item Les 6 racines obtenues sont toujours des \textbf{paires conjuguées} ($\pm \lambda$ ou $\pm a \pm ib$).
    \item Cela valide la symétrie du matériau et la stabilité de l'algorithme de résolution.
\end{itemize}

\hrule
\vspace{0.5cm}

\paragraph{Prochaines étapes possibles (non réalisées) :}
\begin{itemize}
    \item \textbf{Etape 9} : Assemblage de la matrice de transfert globale pour le multicouche.
    \item \textbf{Calcul des Contraintes} : Reconstruction du champ de contraintes complet $\sigma_{ij}(z)$.
\end{itemize}

\end{document}
